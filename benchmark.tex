\section{CNRS Benchmark} \label{sec:benchmark}

The \gls{CNRS} Benchmark \cite{tiberga_results_2020} is a numerical
benchmark for multiphysics software dedicated to modeling \glspl{MSR}. It
consists of three Phases and eight Steps in total. Each
Step is a well-defined subproblem for systematically assessing the
capabilities of \gls{MSR} software and pinpointing sources of discrepancies
between software. Phase 0 consists three single-physics problems in fluid
dynamics, neutronics, and temperature, respectively. Phase 1 consists
of four coupled, steady-state problems. Lastly, Phase 2 consists of one
coupled, time-dependent problem.

\begin{figure}[htb!]
	\centering
	\includegraphics[width=.6\columnwidth]{geometry}
	\caption{2m$\times$2m 2D domain of the CNRS Benchmark. $U_{lid}$
	represents the velocity along the top boundary. Various quantities are
	measured along the centerlines AA' and BB' for comparison. From Tiberga et
	al. \cite{tiberga_results_2020}.}
	\label{fig:geometry}
\end{figure}

As shown in Figure \ref{fig:geometry}, the domain geometry is a 2m$\times$2m
square cavity filled with LiF-BeF$_2$-UF$_4$ molten salt at an initial
temperature of 900K \cite{tiberga_results_2020}.
Standard vacuum boundary conditions apply for neutron flux along all
boundaries whereby outgoing neutrons are considered lost, while homogeneous
boundary conditions apply for delayed neutron precursors. No-slip boundary
conditions apply for velocity variables in the cavity, except along the top
boundary when the subproblem imposes forced flow in the form of lid-driven
cavity flow. For the temperature variable, all boundaries are insulated and we
simulate salt cooling with the following volumetric heat sink equation:
%
\begin{align}
    q'''(\vec{r}) &= \gamma \left(900 - T(\vec{r})\right) \label{eq:heat}
    \shortintertext{where}
    q''' &= \mbox{volumetric heat sink [W$\cdot$m$^{-3}$],}
    \nonumber \\
    \gamma &= \mbox{heat transfer coefficient [W$\cdot$m$^{-3}\cdot$K$^{-1}$],}
    \nonumber \\
    T(\vec{r}) &= \mbox{temperature at point $\vec{r}$ [K].} \nonumber
\end{align}

Tiberga et al. \cite{tiberga_results_2020} used Serpent 2
\cite{leppanen_serpent_2014} with the JEFF-3.1 library
\cite{koning_jeff-31_2006} to generate multigroup neutronics data for the
LiF-BeF$_2$-UF$_4$ salt in the domain at 900K, which they condensed into six
energy groups and eight precursor groups. We direct readers to their paper for
the group constant data \cite{tiberga_results_2020}. In addition, the
benchmark prescribes the following equations to govern the temperature
dependence in the cross sections and the neutron diffusion coefficients:
%
\begin{align}
    \Sigma_i (T) &= \Sigma_i(T_{ref})
    \frac{\rho_{fuel}(T)}{\rho_{fuel}(T_{ref})} \\
    D (T) &= D(T_{ref})
    \frac{\rho_{fuel}(T_{ref})}{\rho_{fuel}(T)}
    \shortintertext{where}
    \Sigma_i &= \mbox{relevant macroscopic cross section [cm${-1}$],}
    \nonumber \\
    D &= \mbox{neutron diffusion coefficient [cm$^2\cdot$s$^{-1}$],}   
    \nonumber \\
    \rho_{fuel} &= \mbox{density of the fuel salt [kg$\cdot$m$^{-3}$],}
    \nonumber \\
    T_{ref} &= \mbox{reference temperature} = 900\mbox{ K}. \nonumber
\end{align}

The benchmark also prescribes incompressible Navier-Stokes flow with the
Boussinesq approximation for evaluating the salt flow in the
domain, but does not restrict the type of neutronics model.
The following subsections briefly detail each benchmark step. Table
\ref{table:benchmark} lists the relevant input parameters and observables for
each step.

\begin{table*}[tp!]
	\caption{Input parameters and observables of each benchmark step.}
	\centering
	\footnotesize
	\begin{tabular}{p{.05\textwidth} p{.35\textwidth} p{.6\textwidth}}
		\toprule
		\textbf{Step} & \textbf{Input parameters} & \textbf{Observables} \\
		\midrule
		0.1 &
		\begin{itemize}[nosep,noitemsep,left=0pt,
		                before={\begin{minipage}[t]{\hsize}},
                        after ={\end{minipage}}]
		    \item $U_{lid} = 0.5$ m$\cdot$s$^{-1}$
		\end{itemize}\vspace*{-\baselineskip}\mbox{} &
		\begin{itemize}[nosep,noitemsep,left=0pt,
		                before={\begin{minipage}[t]{\hsize}},
                        after ={\end{minipage}}]
		    \item Velocity components $(u_x,u_y)$ along AA' and BB'
		\end{itemize}\vspace*{-\baselineskip}\mbox{} \\
        \midrule
        0.2 &
        \begin{itemize}[nosep,noitemsep,left=0pt,
		                before={\begin{minipage}[t]{\hsize}},
                        after ={\end{minipage}}]
		    \item $U_{lid} = 0$ m$\cdot$s$^{-1}$
		    \item $T = 900$ K
		    \item $P = 1$ GW
		\end{itemize} &
		\begin{itemize}[nosep,noitemsep,left=0pt,
		                before={\begin{minipage}[t]{\hsize}},
                        after ={\end{minipage}}]
		    \item Fission rate density $\sum^G_g \Sigma_{f,g} \phi_g(\vec{r})$ along AA'
            \item Reactivity $\rho$
		\end{itemize}\vspace*{-\baselineskip}\mbox{} \\
        \midrule
        0.3 &
        \begin{itemize}[nosep,noitemsep,left=0pt,
		                before={\begin{minipage}[t]{\hsize}},
                        after ={\end{minipage}}]
		    \item Fixed flow field from Step 0.1
		    \item Fixed heat source distribution from Step 0.2
		    \item $\gamma = 10^6$ W$\cdot$m$^{-3}\cdot$K$^{-1}$
		\end{itemize} &
		\begin{itemize}[nosep,noitemsep,left=0pt,
		                before={\begin{minipage}[t]{\hsize}},
                        after ={\end{minipage}}]
		    \item Temperature $T$ along AA' and BB'
		\end{itemize}\vspace*{-\baselineskip}\mbox{} \\
        \midrule
        1.1 &
        \begin{itemize}[nosep,noitemsep,left=0pt,
		                before={\begin{minipage}[t]{\hsize}},
                        after ={\end{minipage}}]
		    \item Fixed flow field from Step 0.1
		    \item $T = 900$ K
		    \item $P = 1$ GW
		\end{itemize} &
		\begin{itemize}[nosep,noitemsep,left=0pt,
		                before={\begin{minipage}[t]{\hsize}},
                        after ={\end{minipage}}]
		    \item Delayed neutron source $\sum_j \lambda_j C_j$ along AA' and BB'
		    \item Reactivity change from Step 0.2 $\Delta \rho = \rho
        - \rho_{s_{0.2}}$
		\end{itemize}\vspace*{-\baselineskip}\mbox{} \\
        \midrule
        1.2 &
        \begin{itemize}[nosep,noitemsep,left=0pt,
		                before={\begin{minipage}[t]{\hsize}},
                        after ={\end{minipage}}]
		    \item Fixed flow field from Step 0.1
		    \item $P = 1$ GW
		    \item $\gamma = 10^6$ W$\cdot$m$^{-3}\cdot$K$^{-1}$
		\end{itemize}\vspace*{-\baselineskip}\mbox{} &
		\begin{itemize}[nosep,noitemsep,left=0pt,
		                before={\begin{minipage}[t]{\hsize}},
                        after ={\end{minipage}}]
		    \item Temperature $T$ along AA' and BB'
            \item Reactivity change from Step 1.1 $\Delta\rho = \rho -
        \rho_{s_{1.1}}$
            \item Change in fission rate density
        $\sum^G_g \Sigma_{f,g} \phi_g(\vec{r}) -
        \left[\sum^G_g \Sigma_{f,g} \phi_g(\vec{r})\right]_{s_{0.2}}$
		\end{itemize} \\
        \midrule
        1.3 &
        \begin{itemize}[nosep,noitemsep,left=0pt,
		                before={\begin{minipage}[t]{\hsize}},
                        after ={\end{minipage}}]
		    \item $P = 1$ GW
		    \item $U_{lid} = 0$ m$\cdot$s$^{-1}$
		    \item $\gamma = 10^6$ W$\cdot$m$^{-3}\cdot$K$^{-1}$
		\end{itemize}\vspace*{-\baselineskip}\mbox{} &
		\begin{itemize}[nosep,noitemsep,left=0pt,
		                before={\begin{minipage}[t]{\hsize}},
                        after ={\end{minipage}}]
		    \item Velocity components $(u_x, u_y)$ along AA' and BB'
            \item Temperature $T$ along AA' and BB'
            \item Delayed neutron source $\sum_j \lambda_j C_j$ along AA' and BB'
            \item Reactivity change from Step 0.2
        $\Delta\rho = \rho - \rho_{s_{0.2}}$
		\end{itemize} \\
        \midrule
        1.4 &
        \begin{itemize}[nosep,noitemsep,left=0pt,
		                before={\begin{minipage}[t]{\hsize}},
                        after ={\end{minipage}}]
		    \item $\gamma = 10^6$ W$\cdot$m$^{-3}\cdot$K$^{-1}$
		    \item $P$ variable in the range $[0,1]$ GW with a step of 0.2 GW
		    \item $U_{lid}$ variable in the range $[0,0.5]$ m$\cdot$s$^{-1}$
		    with a step of 0.1 m$\cdot$s$^{-1}$
		\end{itemize} &
		\begin{itemize}[nosep,noitemsep,left=0pt,
		                before={\begin{minipage}[t]{\hsize}},
                        after ={\end{minipage}}]
		    \item Reactivity change from Step 0.2 $\Delta\rho = \rho
		    - \rho_{s_{0.2}}$
		\end{itemize}\vspace*{-\baselineskip}\mbox{} \\
        \midrule
        2.1 &
        \begin{itemize}[nosep,noitemsep,left=0pt,
		                before={\begin{minipage}[t]{\hsize}},
                        after ={\end{minipage}}]
		    \item $\gamma = 10^6$ W$\cdot$m$^{-3}\cdot$K$^{-1}$
            \item Steady-state solution from Step 1.4 with $U_{lid} = 0.5$
        m$\cdot$s$^{-1}$ and $P = 1.0$ GW
		\end{itemize} &
		\begin{itemize}[nosep,noitemsep,left=0pt,
		                before={\begin{minipage}[t]{\hsize}},
                        after ={\end{minipage}}]
		    \item Power gain and shift as a function of the perturbation frequency
		\end{itemize}\vspace*{-\baselineskip}\mbox{} \\
		\bottomrule
	\end{tabular}
	\label{table:benchmark}
\end{table*}

\subsection{Phase 0: Single physics}

In this preliminary phase, the steady-state solutions of
individual physics are studied without any multiphysics coupling.

\subsubsection{Step 0.1: Velocity field}

This step investigates the steady-state incompressible flow distribution in the
domain from a lid-driven cavity flow by imposing a non-zero horizontal
velocity along the top boundary. In addition, this step provides a fixed
velocity field for Steps 0.3, 1.1, and 1.2.

\subsubsection{Step 0.2: Neutronics}

This step tests neutronics capabilities through a criticality eigenvalue
problem in a static, isothermal fuel configuration by solving for the fission
rate density and effective multiplication factor $k_{eff}$. The total power $P$
is fixed to normalize the neutron fluxes. This step also aims
to identify deviations in results attributable to differences in neutronics
models and approximations. The total power $P$ is fixed to normalize the
neutron fluxes.

\subsubsection{Step 0.3: Temperature}

This step assesses passive scalar transport capability for determining the
temperature distribution independently from
the fluid flow and neutronics problems by imposing fixed velocity and fission
heat source distributions from Steps 0.1 and 0.2, respectively.

\subsection{Phase 1: Steady-state coupling}

Phase 1 builds towards simulating a fully-coupled multiphysics steady-state
system by gradually introducing coupling between various physics present in
a fast-spectrum molten salt system. All simulations are solved as steady-state
criticality eigenvalue problems.

\subsubsection{Step 1.1: Circulating fuel}

This step investigates effects of fuel salt flow on the neutronics,
namely the reactivity loss from the movement of precursors. The delayed neutron
precursors are allowed to drift under the fixed velocity field from Step 0.1
while keeping the temperature $T$ fixed at 900 K. 

\subsubsection{Step 1.2: Power coupling}

This step assesses the capability to accurately reproduce the change in
neutron flux distribution due to the fuel density reactivity feedback between
the neutron fluxes and the temperature distribution. We solve for the
steady-state neutron flux and temperature distributions under the fixed
velocity field from Step 0.1 and a volumetric heat sink described by Equation
\ref{eq:heat}.

\subsubsection{Step 1.3: Buoyancy}

Building on the previous step, this step replaces the fixed velocity field with
buoyancy-driven flow arising from the temperature gradients for a fully-coupled
multiphysics problem without forced flow. Barring any major discrepancy in the
previous steps, this step assesses the capability to reproduce the correct
buoyancy-driven flow profile and the subsequent effects on the neutronics and
temperature distribution due to precursor drift and fuel density reactivity
feedback.

\subsubsection{Step 1.4: Full coupling}

This step introduces forced flow to the fully-coupled problem through the
non-zero $U_{lid}$ boundary
condition. Thus, this problem most closely represents a molten salt system with
1) flow driven by an external force, 2) buoyancy flow effects, 3) \gls{DNP}
drift, and 4) thermal feedback effects on the neutronics. We solve for the
$k_{eff}$ under a range of $U_{lid}$ and $P$ values given in Table
\ref{table:benchmark}.

\subsection{Phase 2: Time dependent coupling}

In this phase, the transient response of the fully coupled nonlinear system is
studied.

\subsubsection{Step 2.1: Forced convection transient}

Linear perturbation analyses are performed in this step by introducing periodic
perturbations to the heat transfer coefficient $\gamma$ and studying the gain
and phase shift of the response in the total power $P$. For the initial
conditions, the steady-state solution from Step 1.4 with
$U_{lid} = 0.5$ m$\cdot$s$^{-1}$ and $P = 1$ GW is used. This initial
configuration is made exactly critical by scaling the neutron source terms,
from fission and \gls{DNP} decay, by the inverse of the criticality eigenvalue
solution from Step 1.4.

$\gamma$ is uniformly perturbed according to small-amplitude sine waves given
as:
\begin{align}
    \gamma =& \gamma_0 \left[ 1 + 0.1\sin\left(2 \pi f \right) \right]
    \shortintertext{where}
    \gamma_0 =& 10^6 \mbox{ W$\cdot$m$^{-3}\cdot$K$^{-1}$}, \nonumber \\
    f \in& \left\lbrace 0.0125, 0.025, 0.05, 0.1, 0.2, 0.4, 0.8 \right\rbrace 
    \mbox{ Hz.} \nonumber
\end{align}

The benchmark defines power gain as:
\begin{align}
    \mbox{Power gain} =& \frac{\left(P_{max} - P_{avg}\right)/P_{avg}}{
    \left(\gamma_{max} - \gamma_{avg}\right)/\gamma_{avg}}
\end{align}
%
The subscripts denote the maximum and time-averaged values of $P$ and $\gamma$.

\FloatBarrier
