%%%%%%%%%%%%%%%%%%%%%%%%%%%%%%%%%%%%%%%%%
% Plain Cover Letter
% LaTeX Template
%
% This template has been downloaded from:
% http://www.latextemplates.com
%
% Original author:
% Rensselaer Polytechnic Institute (http://www.rpi.edu/dept/arc/training/latex/resumes/)
%
%%%%%%%%%%%%%%%%%%%%%%%%%%%%%%%%%%%%%%%%%

%----------------------------------------------------------------------------------------
%       PACKAGES AND OTHER DOCUMENT CONFIGURATIONS
%----------------------------------------------------------------------------------------

\documentclass[11pt]{letter} % Default font size of the document, change to 10pt to fit more text
\usepackage{graphicx}
%\usepackage{newcent} % Default font is the New Century Schoolbook PostScript font
%\usepackage{helvet} % Uncomment this (while commenting the above line) to use the Helvetica font

% Margins
\usepackage[left=1.25in,right=1.25in,top=1in,bottom=0.5in]{geometry}
%\let\raggedleft\raggedright % Pushes the date (at the top) to the left, comment this line to have the date on the right

\usepackage{eso-pic,graphicx}
 \begin{document}
\AddToShipoutPictureBG*{\includegraphics[width=\paperwidth,height=\paperheight]{background.pdf}}

%----------------------------------------------------------------------------------------
%       ADDRESSEE SECTION
%----------------------------------------------------------------------------------------

\begin{letter}{Professor Mostafa Ghiaasiaan\\
Professor Piero Ravetto\\
Professor Sara Pozzi\\
Executive Editors\\
Annals of Nuclear Energy}

%----------------------------------------------------------------------------------------
%       YOUR NAME & ADDRESS SECTION
%----------------------------------------------------------------------------------------

\address{Sun Myung Park\\
University of Illinois\\
226 Talbot Laboratory\\
MC-234\\
104 S. Wright Street\\
Urbana, IL 61801}

\name{
\hspace{3cm} Sun Myung Park\\
\hspace{3cm} Graduate Student\\}


%----------------------------------------------------------------------------------------
%       LETTER CONTENT SECTION
%----------------------------------------------------------------------------------------

\opening{Dear Professors Ghiaasiaan, Ravetto, Pozzi,}

Please find enclosed a manuscript entitled: ``Verification of Moltres for
Multiphysics Simulations of Fast-Spectrum Molten Salt Reactors'' which my
coauthors, Dr Munk and Dr Huff, and I are submitting for exclusive
consideration for publication as a research article in Annals of Nuclear
Energy.

This manuscript describes Moltres, a multiphysics reactor simulation software
built on the Multiphysics Object-Oriented Simulation Environment (MOOSE), and
verifies Moltres' capability to accurately model strongly coupled multiphysics
in fast-spectrum molten salt reactors (MSR). We demonstrate this capability by
accurately reproducing the expected results of the CNRS benchmark, a
multiphysics numerical benchmark designed to assess the physics-coupling
capabilities of MSR simulation software. Through the various subproblems in the
benchmark, our simulation results show consistently good agreement with
results from participating software presented in the CNRS benchmark
paper.

A previous publication introduced Moltres and demonstrated simulations of a
thermal MSR in two and three dimensions. Along with this publication, they
would provide a basis for verifying Moltres' capabilities in modeling MSRs. We
plan to leverage the fully implicit and parallel multiphysics solver and other
advanced features available in Moltres through MOOSE for future work involving
modeling and simulation of fast-spectrum MSRs and other advanced reactors.

Thank you for your consideration of this work. I expect it will be of interest
to a broad readership concerned with multi-physics simulation of molten salt
reactors and other advanced reactors. Please address correspondence concerning
this manuscript to me at smpark3@illinois.edu.

\closing{\hspace{3cm} Sincere regards,}

%----------------------------------------------------------------------------------------

\end{letter}

\end{document}


