\section{Moltres} \label{sec:moltres}

%% Contents
% General description of Moltres and MOOSE
% Group constant data required from SCALE/Serpent
% Neutronics description
% Thermal-hydraulics description

Moltres \citep{lindsay_introduction_2018} is an open-source, \gls{MOOSE}-based
``application'' designed for multiphysics simulations of \glspl{MSR}. The goal
of making Moltres open-source is to promote quality through transparency and
ease of peer review. The source code \citep{lindsay_moltres_2017} is available
on GitHub \citep{github_build_2017}. Moltres leverages on \texttt{git} for
version control, and integrated testing to protect existing capabilities while
concurrently supporting continued code development. Moltres depends on the
\gls{MOOSE} finite element framework for its meshing and parallel, nonlinear
NEWTON-based solver capabilities. Therefore,
Moltres by default has access to tight coupling methods with implicit
time-stepping. Users can also opt to decouple problems for loosely coupled
solves depending on their specific needs. Furthermore, applications within the
\gls{MOOSE} framework share the same programming
interfaces; this commonality simplifies the work required to tightly couple
different physics over a wide range of length and time scales. For \gls{MSR}
simulations in Moltres such as those in this paper, we coupled Moltres'
\gls{MSR} modeling capabilities with \gls{MOOSE}'s \textit{Navier-Stokes} and
\textit{Heat Conduction} physics modules \citep{peterson_overview_2017} for
general thermal-hydraulics modeling. Together with these physics modules,
Moltres solves the multigroup neutron diffusion equations, for an arbitrary
number of energy and precursor groups, and thermal-hydraulics equations
simultaneously on the same mesh.

In a previous work, \cite{lindsay_introduction_2018}
demonstrated Moltres' \gls{MSR} neutronics modeling capabilities with 1D salt
flow in 2D-axisymmetric and 3D models of the \gls{MSRE}. The neutron flux and
temperature distributions showed good qualitative agreement with legacy
\gls{MSRE} data albeit with some minor quantitative discrepancies due to
simplifications and assumptions in the reactor geometry. Moltres has
since undergone further development to support the looping of \gls{DNP} drift
back into the reactor core, coupling the aforementioned \gls{DNP} drift
to incompressible Navier-Stokes velocity flows, and a decay heat model to
simulate decay heat from fission products.

To perform neutronics calculations, Moltres requires homogenized group constant
data from dedicated high-fidelity neutronics software such as SCALE
\citep{dehart_reactor_2011} or Serpent 2 \citep{leppanen_serpent_2014}. Users
can run a Python script in Moltres' Github repository which automatically reads
user-provided SCALE or Serpent 2 output data files and creates
Moltres-compatible JSON or text files containing all required group constant
data. There are on-going efforts to produce a similar script for parsing OpenMC
output data.

Moltres solves for the neutron fluxes governed by
the multigroup neutron diffusion equations given by:
%
\begin{align}
    \frac{1}{v_g} \frac{\partial \phi_g}{\partial t} =& \nabla \cdot D_g
    \nabla \phi_g - \Sigma^r_g \phi_g +
    \sum^G_{g' \neq g} \Sigma^s_{g' \rightarrow g} \phi_{g'} \nonumber \\
    &+ \chi^p_g \sum^G_{g'=1} \left( 1-\beta \right) \nu \Sigma^f_{g'}
    \phi_{g'} + \chi^d_g \sum^I_i \lambda_i C_i, \label{eq:neut} \\
    %
    \intertext{where}
    v_g =& \text{ average speed of neutrons in group $g$ [cm$\cdot$s$^{-1}$],} 
    \nonumber \\
    \phi_g =& \text{ neutron flux in group $g$ [cm$^{-2}\cdot$s$^{-1}$],}
    \nonumber \\
    t =& \text{ time [s],} \nonumber \\
    D_g =& \text{ diffusion coefficient of neutrons in} \nonumber \\
    &\text{ group $g$ [cm$^2\cdot$s$^{-1}$],} \nonumber \\
    \Sigma^r_g =& \text{ macroscopic cross section for removal of} \nonumber \\
    &\text{ neutrons from group $g$ [cm$^{-1}$],} \nonumber \\
    \Sigma^s_{g' \rightarrow g} =& \text{ macroscopic cross section of
    scattering from} \nonumber \\
    &\text{ groups $g'$ to $g$ [cm$^{-1}$],} \nonumber \\
    \chi^p_g =& \text{ prompt fission spectrum for neutrons in} \nonumber \\
    &\text{ group $g$ [ - ],} \nonumber \\
    G =& \text{ total number of discrete neutron groups [ - ],} \nonumber \\
    \nu =& \text{ average number of neutrons produced per} \nonumber \\
    &\text{ fission [ - ],} \nonumber \\
    \Sigma^f_{g} =& \text{ macroscopic fission cross section for neutron}
    \nonumber \\
    &\text{ in group $g$ [cm$^{-1}$],} \nonumber \\
    \chi^d_g =& \text{ delayed fission spectrum for neutrons in} \nonumber \\
    &\text{ group $g$ [ - ],} \nonumber \\
    I =& \text{ total number of delayed neutron precursor} \nonumber \\
    &\text{ groups [ - ],} \nonumber \\
    \beta =& \text{ total delayed neutron fraction [ - ].} \nonumber
\end{align}

While Moltres is generally unit-agnostic, we use the CGS system here because
the length units for the neutron cross section data from SCALE and Serpent 2
are provided in centimeters. The delayed neutron precursor concentrations are
governed by the following equation:
%
\begin{align}
    \frac{\partial C_i}{\partial t} =& \beta_i \sum^G_{g'=1} \nu \Sigma^f_{g'}
    \phi_{g'} - \lambda_i C_i - \vec{u} \cdot \nabla C_i + \nabla \cdot
    D_{\text{P}} \nabla C_i, \label{eq:dnp} \\
    %
    \intertext{where}
    \beta_i =& \text{ delayed neutron fraction of precursor group $i$ [ - ],}
    \nonumber \\
    \lambda_i =& \text{ average decay constant of delayed neutron} \nonumber \\
    &\text{ precursors in precursor group $i$ [s$^{-1}$],} \nonumber \\
    C_i =& \text{ concentration of delayed neutron precursors in}
    \nonumber \\
    &\text{ precursor group $i$ [cm$^{-3}$],} \nonumber \\
    \vec{u} =& \text{ molten salt flow velocity vector [cm$\cdot$s$^{-1}$],}
    \nonumber \\
    D_{\text{P}} =& \text{ effective diffusion coefficient of the delayed}
    \nonumber \\
    &\text{ neutron precursors [cm$^2\cdot$s$^{-1}$].} \nonumber
\end{align}

The last two terms in Equation \ref{eq:dnp} represent the advection and
diffusion terms, respectively, to model the movement of \gls{DNP} in
liquid-fuel \glspl{MSR}.

The governing equation for temperature is an advection-diffusion equation with
a fission heat source term given by:
%
\begin{align}
    \rho c_{p} \frac{\partial T}{\partial t} =& - \rho c_p \vec{u}
    \cdot \nabla T + \nabla \cdot \left(k \nabla T \right) + Q_f
    \label{eq:temp} \\
    %
    \intertext{where}
    Q_f =& \sum^G_{g=1} \epsilon_g \Sigma_g^f \phi_g \nonumber \\
    =& \text{ fission heat source [W$\cdot$m$^{-3}\cdot$s$^{-1}$],} \nonumber
    \\
    \rho =& \text{ density of the molten salt [kg$\cdot$m$^{-3}$],}
    \nonumber \\
    c_p =& \text{ specific heat capacity of molten salt} \nonumber \\
    &\text{ [J$\cdot$kg$^{-1}\cdot$K$^{-1}$],} \nonumber \\
    T =& \text{ temperature of molten salt [K]} \nonumber \\
    k =& \text{ effective thermal conductivity of molten salt} \nonumber \\
    &\text{ [W$\cdot$m$^{-1}\cdot$K$^{-1}$].} \nonumber
\end{align}

Lastly, the governing equations for the incompressible Navier-Stokes flow are
given by:
%
\begin{align}
    \rho \frac{\partial \vec{u}}{\partial t} =&
    -\rho (\vec{u}
    \cdot \nabla) \vec{u} - \nabla p + \mu \nabla^2 \vec{u}
    + \rho \alpha \vec{g} \left(T - T_{\text{ref}} \right)
    \label{eq:momemtum} \\
    %
    \nabla \cdot \vec{u} =& 0
    \label{eq:divergence}
    \intertext{where}
    p =& \text{ pressure [Pa],} \nonumber \\
    \mu =& \text{ dynamic viscosity [Pa$\cdot$s],} \nonumber \\
    \alpha =& \text{ coefficient of thermal expansion [K$^{-1}$],} \nonumber \\
    \vec{g} =& \text{ gravitational force vector [N$\cdot$m$^{-3}$],} \nonumber
    \\
    T_{\text{ref}} =& \text{ reference temperature at which the nominal}
    \nonumber \\
    &\text{ density is provided [K].} \nonumber
    \nonumber
\end{align}

\gls{MOOSE}'s \textit{Navier-Stokes} module also provides stabilization methods
to eliminate numerical node-to-node oscillations commonly observed when
resolving advection-dominated flows using continuous Galerkin methods. We refer
readers to \cite{peterson_overview_2017} for details on the implementation of
these methods in \gls{MOOSE}'s Navier-Stokes module. Moltres handles all the
aforementioned partial differential equations via the continuous finite-element
treatment except for the \gls{DNP} equation which Moltres solves via the
discontinuous finite element method to avoid instabilities in the \gls{DNP}
distribution without the need for stabilization methods.

\subsection{Modeling approach}

% Timestepping, mesh, group constants
% Coupling details for each Step where applicable

For this work, we ran the benchmark cases on a uniformly-spaced mesh consisting
of 200 by 200 elements. Thus, the dimensions of each mesh element are 0.01m by
0.01m. We
approximated most of the relevant variables, i.e. neutron fluxes, velocity
components, pressure, and temperature, using first-order Lagrange shape
functions. The only exception is the precursor concentration variables, which
we approximated using zeroth-order monomial shape functions and solved using
\gls{DFEM}. We took the group constant data directly from
\cite{tiberga_results_2020} and rewrote it into the Moltres-compatible text
format without any other modifications.

As mentioned in Section \ref{sec:moltres}, \gls{MOOSE}'s \textit{Navier-Stokes}
and \textit{Heat Conduction} modules provide some of the capabilities for
modeling incompressible flow and heat transfer. In particular, we stabilized
the incompressible flow and temperature governing equations using the
streamline upwind Petrov-Galerkin and pressure-stabilizing Petrov-Galerkin
stabilization methods \citep{peterson_overview_2017} implemented in
\gls{MOOSE}. Without these stabilization techniques, we observed spurious
numerical oscillations in the velocity and temperature. These oscillations are
commonly reported when using finite element methods to solve
advection-dominated flow and lid-driven cavity problems with corner
singularities where different boundary conditions meet
\citep{kuhlmann_lid-driven_2018}.

We performed all simulations using the Preconditioned \gls{JFNK} solver from
the \gls{MOOSE} framework. The coupled steady-state problems in
\textit{Steps 1.3} and \textit{1.4} required loose coupling between neutronics
and thermal-hydraulics due to the unique problem setup involving an eigenvalue
problem for the neutron multiplication factor and a standard steady-state
problem in thermal-hydraulics simultaneously. All other multiphysics problems
in \textit{Steps} in \textit{Phases 1} and \textit{2} were fully coupled
within the same \gls{JFNK} solve.

For the time-dependent cases in \textit{Step 2.1}, we used a second-order
implicit Backward Differential Formula (BDF2) time-stepping scheme and fixed
the timestep sizes at 1/200 of the period of the corresponding driving
frequencies as shown in Table \ref{table:timestep}. The only exception is the
0.8 Hz case as a timestep size of 0.4s required more than one hundred nonlinear
iterations to converge in each timestep.

\begin{table}[htb!]
    \caption{Timestep sizes used for the time-dependent cases in
    \textit{Step 2}.}
	\centering
	\setlength\tabcolsep{2.5pt}
	\begin{tabular}{l S S S S S S S}
	    \toprule
	    Frequency [Hz] & 0.0125 & 0.025 & 0.05 & 0.1 & 0.2 & 0.4 & 0.8 \\
	    \midrule
	    Timestep size [s] & 0.2 & 0.2 & 0.1 & 0.05 & 0.025 & 0.0125 & 0.00625
	    \\
	    \bottomrule
	\end{tabular}
	\label{table:timestep}
\end{table}