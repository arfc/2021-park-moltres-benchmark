\section{Moltres Model}

Moltres \cite{lindsay_introduction_2018} is an open-source, \gls{MOOSE}-based
``application'' designed for multiphysics simulations of \glspl{MSR}. The goal
of making Moltres open-source is to promote quality through transparency and
ease of peer review. The source code \cite{lindsay_moltres_2017} is available
on GitHub \cite{github_build_2017}. Moltres leverages on \texttt{git} for
version control, and integrated testing to protect existing capabilities while
concurrently supporting continued code development. Moltres depends on the
\gls{MOOSE} finite element framework for its meshing and parallel, nonlinear
NEWTON-based solver capabilities. Therefore,
Moltres by default has access to tight coupling methods with implicit
time-stepping. Users can also opt to decouple problems for loosely coupled
solves depending on their specific needs. Furthermore, applications within the
\gls{MOOSE} framework share the same programming
interfaces; this commonality simplifies the work required to tightly couple
different physics over a wide range of length and time scales. For \gls{MSR}
simulations in Moltres such as those in this paper, we coupled Moltres'
\gls{MSR} modeling capabilities with \gls{MOOSE}'s \textit{Navier-Stokes} and
\textit{Heat Conduction} physics modules \cite{peterson_overview_2017} for
general thermal-hydraulics modeling. Together with these physics modules,
Moltres solves the multigroup neutron diffusion equations, for an arbitrary
number of energy and precursor groups, and thermal-hydraulics equations
simultaneously on the same mesh.

In a previous work, Lindsay et al. \cite{lindsay_introduction_2018}
demonstrated Moltres' \gls{MSR} neutronics modeling capabilities with 1D salt
flow in 2D-axisymmetric and 3D models of the \gls{MSRE}. The neutron flux and
temperature distributions showed good qualitative agreement with legacy
\gls{MSRE} data albeit with some minor quantitative discrepancies due to some
simplifications and assumptions in the reactor geometry. Moltres has
since undergone further development to support the looping of \gls{DNP} drift
back into the reactor core, coupling the aforementioned \gls{DNP} drift
to incompressible Navier-Stokes velocity flows, and a decay heat model to
simulate decay heat from fission products.

\subsection{Neutronics model}

To perform neutronics calculation, Moltres requires homogenized group constant
data from dedicated neutronics software such as SCALE or Serpent 2. Moltres' 
Github repository contains a Python script which reads SCALE or Serpent 2
output text files and rewrites relevant group constant data into a
Moltres-compatible JSON format. There is on-going work to create a similar
script that can parse OpenMC output data.

For the neutron flux, Moltres solves the following multigroup neutron
diffusion:
%


For this study, we coupled Moltres'
neutronics capabilities with \gls{MOOSE}'s incompressible Navier-Stokes
capabilities \cite{peterson_overview_2017} to
model precursor drift and thermal-hydraulics. We performed all simulations
except Step 1.4 using the Preconditioned \gls{JFNK} method with all physics
fully coupled. Step 1.4 required a transient solve on the thermal-hydraulics
portion of the problem to resolve the coupling between lid-driven flow and
buoyancy effects. We then loosely coupled the thermal-hydraulics transient
solve to the neutronics steady-state criticality solve. For future work, we
could explore and determine the appropriate initial conditions or \gls{MOOSE}'s
nonlinear solver settings to circumvent this issue.

For this study, we discretized the problem domain into a 200 by 200 structured
mesh, resulting in uniform mesh elements of dimensions 1cm by 1cm. We
approximated most of the relevant variables, i.e. neutron fluxes, velocity
components, pressure, and temperature, using first-order Lagrange shape
functions. The only exception is the precursor concentration variables, which
we approximated using zeroth-order monomial shape functions and solved using
the Discontinuous Galerkin finite element method to eliminate spurious
numerical oscillations arising from the high Schmidt number flow. As for the
high Reynolds number flow, we stabilized the incompressible Navier-Stokes
equations using the streamline upwind Petrov-Galerkin and pressure-stabilizing
Petrov-Galerkin stabilization methods \cite{peterson_overview_2017} provided in
\gls{MOOSE}.