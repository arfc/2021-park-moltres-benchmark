\section{Moltres} \label{sec:moltres}

In this section, we describe Moltres \cite{lindsay_introduction_2018}, the
multiphysics reactor simulation tool, and the specific modeling approach for
simulating the CNRS Benchmark cases in Moltres. Much of Moltres' development
focuses on meeting the needs of \gls{MSR} multiphysics.

\subsection{Description of Moltres} \label{sec:description-of-moltres}

Moltres models coupled neutronics and thermal-hydraulics in reactors. While
generally applicable to most reactor concepts, much of
Moltres' development focuses on meeting the needs of \gls{MSR} multiphysics.
Together with \gls{MOOSE}'s \cite{permann_moose_2020} \texttt{Heat}
\texttt{Conduction} and \texttt{Navier-Stokes} \cite{peterson_overview_2018}
modules, Moltres solves the multigroup neutron diffusion
equations, for an arbitrary number of energy and precursor groups, and
thermal-hydraulics equations simultaneously on the same mesh (or separately
through fixed-point iterations if desired).
In a previous work, Lindsay et al. \cite{lindsay_introduction_2018}
demonstrated Moltres' \gls{MSR} neutronics modeling capabilities with 1D salt
flow in 2D-axisymmetric and 3D models of the \gls{MSRE}. The neutron flux and
temperature distributions agreed qualitatively with legacy
\gls{MSRE} data albeit with some minor quantitative discrepancies due to
simplifications and assumptions in the reactor geometry. Moltres has
since undergone further development to support 1) the looping of \gls{DNP}
drift back into the reactor core, 2) coupling the aforementioned \gls{DNP}
drift to incompressible Navier-Stokes velocity flows within the reactor core,
and 3) a decay heat model to simulate decay heat from fission products.

To perform neutronics calculations, Moltres requires homogenized group constant
data from dedicated high-fidelity neutronics software such as the NEWT module
in SCALE
\cite{dehart_reactor_2011} or Serpent 2 \cite{leppanen_serpent_2014}. Users
can run a Python script in Moltres' Github repository which automatically reads
user-provided SCALE or Serpent 2 output data files and creates
Moltres-compatible JSON or text files containing all required group constant
data. There are ongoing efforts to produce a similar script for parsing OpenMC
output data.

Moltres solves for the neutron fluxes governed by
the multigroup neutron diffusion equations given by:
%
\begin{align}
    \frac{1}{v_g} \frac{\partial \phi_g}{\partial t} =& \nabla \cdot D_g
    \nabla \phi_g - \Sigma^r_g \phi_g +
    \sum^G_{g' \neq g} \Sigma^s_{g' \rightarrow g} \phi_{g'} \nonumber \\
    &+ \chi^p_g \sum^G_{g'=1} \left( 1-\beta \right) \nu \Sigma^f_{g'}
    \phi_{g'} + \chi^d_g \sum^I_i \lambda_i C_i \label{eq:neut} \\
    %
    \shortintertext{where}
    v_g =& \text{ average speed of neutrons in group $g$,} 
    \nonumber \\
    \phi_g =& \text{ neutron flux in group $g$,}
    \nonumber \\
    t =& \text{ time,} \nonumber \\
    D_g =& \text{ diffusion coefficient of neutrons in} \nonumber \\
    &\text{ group $g$,} \nonumber \\
    \Sigma^r_g =& \text{ macroscopic cross section for removal of} \nonumber \\
    &\text{ neutrons from group $g$,} \nonumber \\
    \Sigma^s_{g' \rightarrow g} =& \text{ macroscopic cross section of
    scattering from} \nonumber \\
    &\text{ groups $g'$ to $g$,} \nonumber \\
    \chi^p_g =& \text{ prompt fission spectrum for neutrons in} \nonumber \\
    &\text{ group $g$,} \nonumber \\
    G =& \text{ total number of discrete neutron groups,} \nonumber \\
    \nu =& \text{ average number of neutrons produced per} \nonumber \\
    &\text{ fission,} \nonumber \\
    \Sigma^f_{g} =& \text{ macroscopic fission cross section for neutron}
    \nonumber \\
    &\text{ in group $g$,} \nonumber \\
    \chi^d_g =& \text{ delayed fission spectrum for neutrons in} \nonumber \\
    &\text{ group $g$,} \nonumber \\
    I =& \text{ total number of delayed neutron precursor} \nonumber \\
    &\text{ groups,} \nonumber \\
    \beta =& \text{ total delayed neutron fraction.} \nonumber
\end{align}

The delayed neutron precursor concentrations are
governed by the following equation:
%
\begin{align}
    \frac{\partial C_i}{\partial t} =& \beta_i \sum^G_{g'=1} \nu \Sigma^f_{g'}
    \phi_{g'} - \lambda_i C_i - \vec{u} \cdot \nabla C_i + \nabla \cdot
    D_{\text{P}} \nabla C_i \label{eq:dnp} \\
    %
    \shortintertext{where}
    \beta_i =& \text{ delayed neutron fraction of precursor group $i$,}
    \nonumber \\
    \lambda_i =& \text{ average decay constant of delayed neutron} \nonumber \\
    &\text{ precursors in precursor group $i$,} \nonumber \\
    C_i =& \text{ concentration of delayed neutron precursors in}
    \nonumber \\
    &\text{ precursor group $i$,} \nonumber \\
    \vec{u} =& \text{ molten salt flow velocity vector,}
    \nonumber \\
    D_{\text{P}} =& \text{ effective diffusion coefficient of the delayed}
    \nonumber \\
    &\text{ neutron precursors.} \nonumber
\end{align}

The last two terms in Equation \ref{eq:dnp} represent the advection and
diffusion terms, respectively, to model the movement of \gls{DNP} in
liquid-fuel \glspl{MSR}.

The governing equation for temperature is an advection-diffusion equation with
a fission heat source term given by:
%
\begin{align}
    \rho c_{p} \frac{\partial T}{\partial t} =& - \rho c_p \vec{u}
    \cdot \nabla T + \nabla \cdot \left(k \nabla T \right) + Q_f
    \label{eq:temp} \\
    %
    \shortintertext{where}
    Q_f =& \sum^G_{g=1} \epsilon_g \Sigma_g^f \phi_g \nonumber \\
    =& \text{ fission heat source,} \nonumber
    \\
    \rho =& \text{ density of the molten salt,}
    \nonumber \\
    c_p =& \text{ specific heat capacity of molten salt,} \nonumber \\
    T =& \text{ temperature of molten salt,} \nonumber \\
    k =& \text{ effective thermal conductivity of molten salt.} \nonumber \\
\end{align}

Lastly, the governing equations for the incompressible Navier-Stokes flow are
given by:
%
\begin{align}
    \rho \frac{\partial \vec{u}}{\partial t} =&
    -\rho (\vec{u}
    \cdot \nabla) \vec{u} - \nabla p + \mu \nabla^2 \vec{u}
    + \rho \alpha \vec{g} \left(T - T_{\text{ref}} \right)
    \label{eq:momemtum}
    \shortintertext{and}
    \nabla \cdot \vec{u} =& 0
    \label{eq:divergence}
    \shortintertext{where}
    p =& \text{ pressure,} \nonumber \\
    \mu =& \text{ dynamic viscosity,} \nonumber \\
    \alpha =& \text{ coefficient of thermal expansion,} \nonumber \\
    \vec{g} =& \text{ gravitational force vector,} \nonumber
    \\
    T_{\text{ref}} =& \text{ reference temperature at which the nominal}
    \nonumber \\
    &\text{ density is provided.} \nonumber
    \nonumber
\end{align}

The velocity, temperature, and delayed neutron precursor
variables are all susceptible to numerical node-to-node oscillations
commonly observed when resolving advection-dominated transport using continuous
Galerkin methods \cite{kuhlmann_lid-driven_2018}.
\gls{MOOSE}'s \texttt{Navier-Stokes} module provides the
\gls{SUPG} stabilization scheme \cite{brooks_streamline_1982} for the velocity
and temperature variables to combat these oscillations. We
refer readers to \cite{peterson_overview_2018} for details on the
implementation of these methods in the \texttt{Navier-Stokes} module. On the
other hand, for the delayed neutron precursor variables,
we discretized them using
discontinuous shape functions supported by \gls{MOOSE}'s discontinuous finite
element solver to circumvent the numerical instability issue.

\subsection{Modeling approach} \label{sec:model}

For this work, we ran the benchmark cases on a uniformly-spaced mesh consisting
of 200$\times$200 elements. Thus, the dimensions of each mesh element are
0.01m$\times$0.01m. For nuclear data, we converted the group constant data
provided by Tiberga et al.
\cite{tiberga_results_2020} into a Moltres-compatible text format. Next, we
discretized most of the relevant variables, i.e. neutron fluxes, velocity
components, pressure, and temperature, using continuous, first-order Lagrange
shape functions. The only exception is the precursor concentration variables,
which we discretized using zeroth-order monomial shape functions and solved
using \gls{DFEM}. We interpolated the resulting discontinuous,
cell-centered precursor values to obtain the nodal values for results
analysis.

As mentioned in Section \ref{sec:description-of-moltres}, the
\texttt{Navier-Stokes} and \texttt{Heat} \texttt{Conduction} modules from \gls{MOOSE}
provide some of the capabilities for
modeling incompressible flow and heat transfer. In particular, we stabilized
the incompressible flow and temperature governing equations using the
\gls{SUPG} stabilization method implemented in \gls{MOOSE}
\cite{peterson_overview_2018}. Without \gls{SUPG} stabilization, we
observed spurious numerical oscillations in the velocity and temperature near
the top boundary due to the singularity on the top left corner where different
velocity boundary conditions meet. We also applied the \gls{PSPG} stabilization
scheme \cite{hughes_new_1986} from the Navier-Stokes module
\cite{peterson_overview_2018}
which enables equal-order discretizations in the velocity and pressure
variables. Equal-order discretizations with \gls{PSPG} are computationally
cheaper and more convenient to work with than implementing higher-order
velocity discretizations for stability without \gls{PSPG}
\cite{chapelle_inf-sup_1993}.

\begin{table}[tb]
    \caption{Timestep sizes used for the time-dependent cases in
    Step 2.}
	\centering
	\setlength\tabcolsep{2.5pt}
	\begin{tabular}{l l l l l l l l}
	    \toprule
	    Frequency [Hz] & 0.0125 & 0.025 & 0.05 & 0.1 & 0.2 & 0.4 & 0.8 \\
	    \midrule
	    Timestep size [s] & 0.2 & 0.2 & 0.1 & 0.05 & 0.025 & 0.0125 & 0.00625
	    \\
	    \bottomrule
	\end{tabular}
	\label{table:timestep}
\end{table}

We performed all eigenvalue calculations in Steps 0.2, 1.1, 1.2, 1.3, and 1.4
using the inverse power method solver in \gls{MOOSE}. All other Steps
were performed using the Preconditioned Newton-Krylov solver
\cite{gaston_physics-based_2015}. The coupled steady-state problems in
Steps 1.2, 1.3, and 1.4 required segregated solvers for the neutronics
and the thermal-hydraulics due to the unique problem
setups involving an eigenvalue problem for the neutron multiplication factor
and a steady-state problem in thermal-hydraulics simultaneously.

For the time-dependent cases in Step 2.1, we employed fully coupled solves with
a second-order implicit Backward Differential Formula (BDF2) time-stepping
scheme. For each driving frequency in the heat transfer coefficient, we set the
timestep sizes to 1/200th of the perturbation period. Table
\ref{table:timestep} shows the timestep sizes for clarity. We assumed the
systems reached asymptotic behavior when the magnitudes of neighboring power
peaks differed by less than 0.001\% for at least ten wavelengths. Under this
assumption, the phase shift measurements between adjacent waves always
converged before the magnitude measurements of the power peaks.

Table \ref{table:software} compares the numerical methods, meshing schemes, and
neutronics models of Moltres and the four participating software packages in
the \gls{CNRS} benchmark paper \cite{tiberga_results_2020}. The $SP_N$ and
$S_N$ neutronics models refer to the simplified $P_N$ spherical harmonics and
$S_N$ discrete ordinates neutron transport models, respectively. Based on the
solvers and methods of solution, Moltres is most similar to the
PHANTOM-$S_N$ + DGFlows \cite{tiberga_discontinuous_2019} multiphysics package
from \gls{TUD} with the $S_2$ neutron transport model. Participants from
\gls{CNRS} and \gls{PSI}
employed non-uniform meshes which were refined near the boundaries while we and
the participants from \gls{PoliMi} and \gls{TUD} employed uniform meshes.

\FloatBarrier

\begin{landscape}
\begin{table*}[p]
    \caption{List of software packages and their corresponding model
    specifications for the CNRS Benchmark simulations
    \cite{tiberga_results_2020}.}
    \centering
    \begin{tabular}{p{4.2cm} p{7cm} p{3.3cm} p{2cm} p{2.7cm}}
        \toprule
        Software & Institute & Numerical method & Mesh & Neutronics model \\
        \midrule
        OpenFOAM & Centre national de la recherche scientifique (CNRS) & Finite volume & 200$\times$200 \newline Non-uniform & $SP_1$ \& $SP_3$ \\
        OpenFOAM & Politecnico di Milano (PoliMi) & Finite volume & 400$\times$400 \newline Uniform & Neutron diffusion \\
        GeN-Foam & Paul Scherrer Institute (PSI) & Finite volume & 200$\times$200 \newline Non-uniform & Neutron diffusion \\
        PHANTOM-$S_N$+DGFlows & Delft University of Technology (TUD) & Discontinuous finite \newline element & 50$\times$50 \newline Uniform & $S_2$ \& $S_6$ \\
        Moltres (This work) & University of Illinois at Urbana-Champaign (UIUC) & Continuous \& discontinuous finite element & 200$\times$200 \newline Uniform & Neutron diffusion \\
        \bottomrule
    \end{tabular}
    \label{table:software}
\end{table*}
\end{landscape}

\FloatBarrier