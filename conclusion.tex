\section{Conclusions}

\glspl{MSR} feature significant multiphysics interactions which present
computational challenges for many existing multiphysics reactor analysis
software. This paper presents code-to-code verification of Moltres
capabilities in modeling such multiphysics phenomena in fast-spectrum
\glspl{MSR} based on the CNRS benchmark \cite{tiberga_results_2020}.
The CNRS benchmark assesses multiphysics \gls{MSR} simulation
software through several steps involving single-physics and coupled
neutronics/thermal-hydraulics problems.

The results showed that Moltres is consistent with the participating software
presented in the CNRS benchmark paper for the modeling of important phenomena
in fast-spectrum \glspl{MSR}. The percentage discrepancies in the various
neutronics, velocity, and temperature quantities mostly fall below or within
one standard deviation of the average of the benchmark participants.
Minor deviations in the temperature in Steps 0.3 and 1.2 
stem from the discontinuous velocity
boundaries on the top corners in the lid-driven cavity flow. We have shown that
these deviations are limited to the top boundary of the domain and do not
affect the rest of the physical parameters. The results from
Moltres agree closest with the TUD-S$_2$ software package, which implements the
$S_2$ discrete ordinates method for
neutron transport on a uniform structured mesh with a \gls{DFEM}-based solver.
These features make Moltres the most similar to the TUD-$S_2$ model as compared
to the other models which employ different neutron transport models,
non-uniform meshes, and/or finite volume-based solvers.

This work verifies Moltres' capabilities for future work involving modeling and
simulation of fast-spectrum \glspl{MSR}. Fast-spectrum \glspl{MSR}
under consideration for modeling with Moltres include the European \gls{MSFR}
as a continuation of work done in \cite{park_advancement_2020}, and
TerraPower's \gls{MCFR} \cite{terrapower_terrapower_2021} from publicly
available design specifications. Moltres can play an important role in
supporting further \gls{MSR} development through enabling transient accident
safety analysis and design optimization studies on an open-source platform.
An ongoing research project involves employing Moltres as a
surrogate model for machine learning-based reactor design optimization.
We note that Moltres also supports modeling solid-fueled reactors such as the
\gls{HTGR} by disabling the precursor drift functionality as demonstrated by
\cite{fairhurst-agosta_multi-physics_2020}. Future work pertaining to
further Moltres development include introducing an intermediate-fidelity
turbulence model for highly turbulent flows in \glspl{MSR}, improving
neutronics accuracy in heterogenous geometries, and enhancing the general
computational performance of existing features.

\FloatBarrier
