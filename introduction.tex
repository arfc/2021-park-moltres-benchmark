\section{Introduction} \label{sec:intro}

\glspl{MSR} have attracted increasing interest over the past two
decades with various on-going research programs funding \gls{MSR} development
in the EU \citep{cordis_severe_nodate}, China \citep{dai_17_2017}, and the US
\citep{doe_office_2021}. These programs include efforts to develop
reactor analysis software specifically tuned for \glspl{MSR}. Reactor
analysis software inform design choices aligned with
maximizing safety, reducing proliferation risks, and improving fuel
efficiency in the next generation of advanced reactors.

Liquid-fueled \glspl{MSR} present new challenges in computational reactor
safety analysis arising from fuel movement. \glspl{MSR} feature strong
negative temperature reactivity feedback in the primary fuel salt. The feedback
causes strong and near-instantaneous
interactions between reactor power and thermal-hydraulics. Thus,
unexpected changes in coolant flow and temperature greatly affect reactor
power and vice versa. Additionally, \gls{MSR} simulation software must include
capabilities to model the movement of delayed neutron precursors
and heat generation in the fuel salt; these physical phenomena are
absent in solid-fuel reactors. The movement of delayed neutron precursors
impacts the effective delayed neutron fraction in the core and consequently
also impacts the transient behavior of the reactor. At the same time, the fuel
salt serves a dual role of generating fission heat and cooling the reactor
core, leading to stronger coupling between heat generation and salt flow
dynamics than traditional \glspl{LWR}.

In recent years, several simulation tools have been developed for full-core
modeling of fast-spectrum \glspl{MSR}. \textit{Tightly coupled} approaches
through segregated solvers involve coupling separate single-physics neutronics
and thermal-hydraulics software. For example, researchers at
the Delft University of Technology coupled 3D neutron diffusion software
DALTON \citep{boer_validation_2010} and CFD software HEAT
\citep{de_zwaan_static_2007} to perform a safety analysis of the \gls{MSFR}
\citep{fiorina_modelling_2014}. In a later effort from the same institute,
Tiberga et al. \cite{tiberga_discontinuous_2019} coupled PHANTOM-$S_N$ and
DGFlows in their participation in the
\gls{CNRS} Benchmark study \citep{tiberga_results_2020}. Another multiphysics
package was developed at the Paul Scherrer Institute (PSI) coupling
thermal-hydraulics system software \gls{TRACE} \citep{nrc_trace_2007} with the
nodal neutron diffusion software \gls{PARCS} \citep{downar_parcs_2010} for the
safety analysis of the \gls{MSFR} \citep{pettersen_coupled_2016}. Coupling
single-physics software to form an integrated multiphysics tool allows
researchers to leverage on older, well-validated single-physics software.
These single-physics software are also highly optimized for solving specific
types of \glspl{PDE} relevant to the investigated system.

With modern
advancements in computing hardware and growing access to high-performance
computing, others \citep{laureau_transient_2017,blanco_neutronic_2020} have
developed multiphysics tools by coupling \gls{CFD} software OpenFOAM
\cite{openfoam_openfoam_2021} with Monte Carlo particle transport software
Serpent 2 \cite{leppanen_serpent_2014}, thus achieving high-fidelity neutronics
calculations in transient reactor analyses. Laureau et al.
\cite{laureau_transient_2017} developed an innovative technique called the
\gls{TFM} method through the introduction of additional time-dependence
operators to conventional fission matrices typically used to accelerate source
convergence in Monte Carlo neutronics solves. The \gls{TFM} method
uses Serpent 2 to pre-calculate three \glspl{TFM} of the reactor system and
interpolating the matrices during the actual transient calculations to
incorporate the effects of temperature-induced density change and Doppler
effect on the neutron cross sections and ultimately the neutron flux. Blanco
\cite{blanco_neutronic_2020} took a more computationally intensive approach of
compiling Serpent 2 as an internal \texttt{C}-based function within OpenFOAM's
\texttt{C++}-based framework. This approach reduced the amount of required data
transfers between Serpent 2 and OpenFOAM as both software have access to shared
memory during runtime. Their integrated tool employs the Quasi-Static
method for transient neutronics calculations and runs Serpent 2 Monte Carlo
calculations several times per timestep until convergence is reached.

Another \gls{MSR} simulation approach involves
developing inherently multiphysics software which perform both
neutronics and thermal-hydraulics calculations. Among earlier efforts,
\cite{nicolino_coupled_2008} and \cite{zhang_development_2009} recognized the
need for more robust multiphysics coupling techniques and higher-fidelity
thermal hydraulics solutions to accurately capture complex flow profiles in
pool-type \glspl{MSR}. They each independently developed unnamed multiphysics
simulation tools and demonstrated their tools with non-moderated \gls{MSR}
designs. Later, \cite{li_transient_2015} demonstrated the steady-state and
transient analysis capabilities of COUPLE, a neutronics and thermal-hydraulics
software developed at the Karlsruhe Institute of Technology. Others adopted
general multiphysics software platforms such as OpenFOAM
\cite{openfoam_openfoam_2021} and \gls{MOOSE} \cite{gaston_physics-based_2015}
to develop multiphysics reactor analysis software. Researchers at Politecnico
di Milano developed a \gls{MSR} simulation tool in COMSOL
\cite{comsol_ab_comsol_2018} and modeled the \gls{MSBR} as a single
axisymmetric fuel channel with a uniform flow profile
\cite{cammi_multi-physics_2011}, followed by the \gls{MSRE} core also as a
single axisymmetric fuel channel with parabola-shaped laminar flow
\cite{cammi_dimensional_2012}. They expanded on their approach by modeling the
\gls{MSRE} upper plenum, downcomer and lower plenum, primary heat exchanger,
and secondary heat exchanger as 0D systems (lumped-parameter model), and
substituting the 2D fuel channel with a 3D fuel channel which more closely
resembled the actual fuel channels in the \gls{MSRE}
\cite{zanetti_geometric_2015}. Beyond graphite-moderated \glspl{MSR}, they've
also modeled the \gls{MSFR} in the same publication which featured Delft
University of Technology's DALTON + HEAT coupled multiphysics tool described
above. More recently, several European institutes (\gls{CNRS}, Politecnico di
Milano, and \gls{PSI}) have also developed coupled neutronics and
thermal-hydraulics tools in OpenFOAM. Their tools share some
similarities such as implementing the $SP_N$ simplified $P_N$ neutron transport
model and leveraging OpenFOAM's turbulent flow modeling capabilities.
Differences include the aforementioned external coupling capability with
Serpent 2 from \gls{CNRS} \cite{blanco_neutronic_2020}, fuel
compressibility modeling and helium bubble tracking capability from Politecnico
di Milano \cite{cervi_development_2019}, and fuel performance analysis
capability from \gls{PSI} \cite{fiorina_creation_2018}.

Finally, within the open-source, finite-element, multiphysics framework, MOOSE,
simulation tools capable of modeling
\glspl{MSR} include Rattlesnake \cite{wang_rattlesnake_2021}, and Moltres
\cite{lindsay_moltres_2017}, the subject of this work.
Rattlesnake primarily tackles radiation transport problems, but the MOOSE
framework facilitates multiphysics coupling
with other MOOSE-based applications for thermal-hydraulics
such that all applications share the same data structure, thus eliminating
computational costs from external data transfers and optionally allowing for
\textit{fully coupled} solves in which all physics to be solved simultaneously
instead of sequentially through fixed-point iterations in tightly coupled
solves. Similarly, Moltres benefits from the highly-integrated
cross-compatibility
within the ecosystem of MOOSE-based applications. Abou-Jaoude et al.
\cite{abou-jaoude_coupled_2020} coupled Rattlesnake with Pronghorn, another
MOOSE-based application for advanced reactor thermal-hydraulics modeling, to
demonstrate several steady-state \gls{MSR} simulation capabilities defined in
the \gls{CNRS} benchmark. Lindsay et al.
\cite{lindsay_introduction_2018} first demonstrated Moltres' \gls{MSR} modeling
capabilities on 2D axisymmetric and 3D Cartesian models of the \gls{MSRE} with
fixed velocity flow on a fully coupled neutronics and thermal-hydraulics solve.
We later demonstrated some of Moltres' more recent developments through
modeling a 2D axisymmetric model of the \gls{MSFR} for steady-state operation
and transient accident analysis \cite{park_advancement_2020}. The latter study
introduced looped \gls{DNP} flow, coupling the \gls{DNP} drift and temperature 
advection-diffusion to incompressible flow, and decay heat modeling
capabilities.

This paper presents benchmarking results from Moltres, an open-source
multiphysics simulation software for advanced reactors. Making Moltres
open-source promotes quality and participation through transparency and
ease of peer review. The source code \citep{lindsay_moltres_2017} is available
on GitHub \citep{github_build_2017}. Moltres leverages \texttt{git} for
version control, and integrated testing to protect existing capabilities while
concurrently supporting continued code development. Moltres depends on the
\gls{MOOSE} finite element framework for its meshing and parallel, nonlinear
Newton-Krylov solver capabilities. Therefore Moltres, by default, has access to
\textit{fully coupled} methods with implicit time-stepping.
\textit{Full coupling} in the context of numerical methods refers to
solving multiple physics represented by multiple equations simultaneously.
Users also have the flexibility of separating different physics through
\textit{tight coupling} in which multiphysics coupling is achieved by
fixed-point iterations. The combination of robust and flexible coupling
methods extends Moltres' capabilities for tackling various phenomena in
nuclear reactors from prompt neutron responses at microsecond time scales to
\gls{DNP} advection-diffusion and thermal-hydraulics at larger time scales.

The mutual compatibility among different physics applications within the
\gls{MOOSE} framework simplifies the work required to strongly couple
different physics together to solve novel multiphysics problems. For \gls{MSR}
simulations in Moltres such as those in this study, we coupled Moltres'
\gls{MSR} modeling capabilities with \gls{MOOSE}'s \texttt{Navier-Stokes} and
\texttt{Heat Conduction} physics modules \cite{peterson_overview_2018} for
general thermal-hydraulics modeling.

The benchmark of interest in this work is the CNRS Benchmark
\citep{tiberga_results_2020} for fast-spectrum \gls{MSR} simulation tools.
Software benchmarks crucially provide a common basis of comparison among
software tools designed
to solve similar computational problems within the same application areas.
Rigorous code-to-code verification of reactor physics software has
important implications for reactor safety and the risks of high
radioactivity exposure in nuclear accident scenarios.
