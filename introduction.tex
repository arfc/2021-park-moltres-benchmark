\section{Introduction} \label{sec:intro}

\glspl{MSR} have attracted increasing interest over the past two
decades with various on-going research programs funding \gls{MSR} development
in the EU \cite{cordis_severe_nodate}, China \cite{dai_17_2017}, and the US
\cite{doe_office_2021}. These programs include efforts to develop
reactor analysis software specifically tuned for \glspl{MSR}. Reactor
analysis software inform design choices aligned with
maximizing safety, reducing proliferation risks, and improving fuel
efficiency in the next generation of advanced reactors.

Liquid-fueled \glspl{MSR} present new challenges in computational reactor
safety analysis arising from fuel movement. \glspl{MSR} feature strong
negative temperature reactivity feedback in the primary fuel salt. The feedback
causes strong and near-instantaneous
interactions between reactor power and thermal-hydraulics. Thus,
unexpected changes in coolant flow and temperature greatly affect reactor
power and vice versa and this requires tighter coupling between these physics
in the software. Additionally, \gls{MSR} simulation software must include
capabilities to model the movement of delayed neutron precursors
and heat generation in the fuel salt; these physical phenomena are
absent in solid-fuel reactors. The movement of delayed neutron precursors
impacts the effective delayed neutron fraction in the core and consequently
also impacts the transient behavior of the reactor. At the same time, the fuel
salt serves a dual role of generating fission heat and cooling the reactor
core, leading to stronger coupling between heat generation and salt flow
dynamics than traditional \glspl{LWR}.

In recent years, several simulation tools have been developed for full-core
modeling of fast-spectrum \glspl{MSR}. \textit{Tightly coupled} approaches
through segregated solvers involve coupling separate single-physics neutronics
and thermal-hydraulics software. For example, researchers at
the \gls{TUD} coupled the 3D neutron diffusion software DALTON
\cite{boer_validation_2010} and the CFD software HEAT
\cite{de_zwaan_static_2007} to perform a safety analysis of the \gls{MSFR}
\cite{fiorina_modelling_2014}. In a later effort from the same institute,
Tiberga et al. \cite{tiberga_discontinuous_2019} coupled PHANTOM-$S_N$ and
DGFlows in their participation in the CNRS benchmark study
\cite{tiberga_results_2020}. The CNRS benchmark, named after the \gls{CNRS}
where it was originally developed, facilitates code-to-code verification of
\gls{MSR} multiphysics software \cite{aufiero_testing_2018}. Another
multiphysics package was developed at the Paul Scherrer Institute (PSI)
coupling the thermal-hydraulics system software \gls{TRACE}
\cite{nrc_trace_2007} with the
nodal neutron diffusion software \gls{PARCS} \cite{downar_parcs_2010} for the
safety analysis of the \gls{MSFR} \cite{pettersen_coupled_2016}. Coupling
single-physics software to form an integrated multiphysics tool allows
researchers to leverage on older, well-validated single-physics software.
These single-physics software are also highly optimized for solving specific
types of \glspl{PDE} relevant to the investigated system.

With modern advancements in computing hardware and growing access to
high-performance computing systems, others have
developed multiphysics tools by coupling the \gls{CFD} software OpenFOAM
\cite{openfoam_openfoam_2021} with the Monte Carlo particle transport software
Serpent 2 \cite{leppanen_serpent_2014}, thus achieving high-fidelity neutronics
calculations in transient reactor analyses. Laureau et al.
\cite{laureau_transient_2017} developed an innovative technique called the
\gls{TFM} method through the introduction of additional time-dependence
operators to conventional fission matrices typically used to accelerate source
convergence in Monte Carlo neutronics solves. The \gls{TFM} method
pre-calculates three \glspl{TFM} of the reactor system in Serpent 2 and
interpolates the matrix values during the actual transient calculations to
incorporate the effects of temperature-induced density change and Doppler
effect on the neutron cross sections and ultimately the neutron flux. Blanco
\cite{blanco_neutronic_2020} took a more computationally intensive approach of
compiling Serpent 2 as an internal \texttt{C}-based function within OpenFOAM's
\texttt{C++}-based framework. This approach reduced the amount of required data
transfers between Serpent 2 and OpenFOAM as both software have access to shared
memory during runtime. Their integrated tool employs the Quasi-Static
method for transient neutronics calculations and runs Serpent 2 Monte Carlo
calculations several times per timestep until convergence is reached.

Another \gls{MSR} simulation approach involves developing ``all-in-one''
multiphysics software which handle all multiphysics calculations and data
transfer internally. Among earlier efforts, Nicolino et al.
\cite{nicolino_coupled_2008} and Zhang et al. \cite{zhang_development_2009}
recognized the
need for more robust multiphysics coupling techniques and higher-fidelity
thermal hydraulics solutions to accurately capture complex flow profiles in
pool-type \glspl{MSR}. They each independently developed unnamed multiphysics
simulation tools and demonstrated their tools with non-moderated \gls{MSR}
designs. Later, Li et al. \cite{li_transient_2015} demonstrated the
steady-state and transient analysis capabilities of COUPLE, a neutronics and
thermal-hydraulics software developed at the Karlsruhe Institute of Technology.
Others adopted extensible software frameworks for developing numerical solvers
to develop multiphysics reactor analysis software. Examples of these software
frameworks include the commercial COMSOL Multiphysics software
\cite{comsol_ab_comsol_2018}, the aforementioned open-source CFD toolbox
OpenFOAM \cite{openfoam_openfoam_2021}, and the open-source finite-element
framework \gls{MOOSE} \cite{gaston_physics-based_2015}. Researchers at
\gls{PoliMi} developed a \gls{MSR} simulation tool in COMSOL and
modeled the \gls{MSBR} as a single axisymmetric fuel channel with a uniform
flow profile \cite{cammi_multi-physics_2011}, followed by the \gls{MSRE} core
also as a single axisymmetric fuel channel with parabola-shaped laminar flow
\cite{cammi_dimensional_2012}. They later expanded on their approach by
modeling the \gls{MSRE} upper plenum, downcomer and lower plenum, primary heat
exchanger, and secondary heat exchanger as 0D systems (lumped-parameter model),
and substituting the 2D fuel channel with a 3D fuel channel which more closely
resembled the actual fuel channels in the \gls{MSRE}
\cite{zanetti_geometric_2015}. Beyond graphite-moderated \glspl{MSR}, they've
also modeled the \gls{MSFR} in the same publication which featured \gls{TUD}'s
DALTON + HEAT coupled multiphysics tool described
above. More recently, several European institutes (\gls{CNRS}, \gls{PoliMi},
and \gls{PSI}) have also developed coupled neutronics and
thermal-hydraulics tools in OpenFOAM. Their tools share some
similarities such as implementing the $SP_N$ simplified $P_N$ neutron transport
model and leveraging OpenFOAM's turbulent flow modeling capabilities.
Differences include fuel compressibility modeling and helium bubble tracking
capabilities from \gls{PoliMi} \cite{cervi_development_2019}, fuel
performance analysis capability from \gls{PSI} \cite{fiorina_creation_2018},
and the aforementioned external coupling capability with Serpent 2 from
\gls{CNRS} \cite{blanco_neutronic_2020}.

Finally, within the open-source, finite-element, multiphysics framework, MOOSE,
simulation tools capable of modeling
\glspl{MSR} include: Rattlesnake \cite{wang_rattlesnake_2021}; and Moltres
\cite{lindsay_moltres_2017}, the subject of this work.
Rattlesnake primarily tackles radiation transport problems, but the MOOSE
framework facilitates multiphysics coupling
with other MOOSE-based applications for other physics
such that all applications share the same data structure. This eliminates
computational costs from external data transfers and optionally allowing for
\textit{fully coupled} solves in which the application solves all physics
simultaneously. Similarly, Moltres benefits from the highly-integrated
cross-compatibility
within the ecosystem of MOOSE-based applications. Abou-Jaoude et al.
\cite{abou-jaoude_coupled_2020} coupled Rattlesnake with Pronghorn, another
MOOSE-based application for advanced reactor thermal-hydraulics modeling, to
demonstrate several steady-state \gls{MSR} simulation capabilities defined in
the CNRS benchmark. Lindsay et al.
\cite{lindsay_introduction_2018} first demonstrated Moltres' \gls{MSR} modeling
capabilities on 2D axisymmetric and 3D Cartesian models of the \gls{MSRE} with
fixed velocity flow on a fully coupled neutronics and thermal-hydraulics solve.
We later demonstrated some of Moltres' more recent developments through
modeling a 2D axisymmetric model of the \gls{MSFR} for steady-state operation
and transient accident analysis \cite{park_advancement_2020}. The latter study
introduced looped \gls{DNP} flow, coupling the \gls{DNP} drift and temperature 
advection-diffusion to incompressible flow, and decay heat modeling
capabilities.

The mutual compatibility among different physics applications within the
\gls{MOOSE} framework simplifies the work required to strongly couple
different physics together to solve novel multiphysics problems. For \gls{MSR}
simulations in Moltres such as those in this study, we coupled Moltres'
\gls{MSR} modeling capabilities with \gls{MOOSE}'s \texttt{Navier-Stokes} and
\texttt{Heat} \texttt{Conduction} physics modules \cite{peterson_overview_2018}
for general thermal-hydraulics modeling.

This paper presents benchmarking results from Moltres, an open-source
multiphysics simulation software for advanced reactors. Making Moltres
open-source promotes quality and participation through transparency and
ease of peer review. The source code \cite{lindsay_moltres_2017} is available
on GitHub \cite{github_build_2017}. Moltres leverages \texttt{git} for
version control, and integrated testing to protect existing capabilities while
concurrently supporting continued code development. Moltres depends on the
\gls{MOOSE} finite element framework for its meshing and parallel, nonlinear
Newton-Krylov solver capabilities. Therefore Moltres, by default, has access to
\textit{fully coupled} methods with implicit time-stepping.
\textit{Full coupling} in the context of numerical methods refers to
solving multiple physics represented by multiple equations simultaneously.
Users also have the flexibility of separating different physics through
\textit{tight coupling} in which multiphysics coupling is achieved through
fixed-point iterations. The combination of robust and flexible coupling
methods extends Moltres' capabilities for tackling various phenomena in
nuclear reactors from prompt neutron responses at microsecond time scales to
\gls{DNP} advection-diffusion and thermal-hydraulics at larger time scales.
