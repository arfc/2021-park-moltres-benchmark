\section{Introduction}

\glspl{MSR} have attracted increasing research interest over the past two
decades. Some of the
current research programs funding \gls{MSR} development include the
\gls{SAMOSAFER} project in the EU \citep{cordis_severe_nodate}, the TMSR
program in China \citep{dai_17_2017}, and the \gls{ARDP} in the US
\citep{department_of_energy_office_2021}. These programs include
efforts towards developing
reactor analysis software specifically tuned for \glspl{MSR}. Reactor
analysis software are important tools in reactor development because
they help inform design choices in line with overarching goals, such as
maximizing safety, reducing proliferation risks, and improving fuel
efficiency, for the latest generation of advanced reactors.

Liquid-fuel \glspl{MSR} present new challenges in computational reactor
safety analysis arising from the liquid fuel form. \glspl{MSR} feature strong
negative reactivity feedback in the primary coolant which holds the dissolved
fissile material. The feedback causes strong and near-instantaneous
interactions between reactor power and thermal-hydraulics. Thus,
unexpected changes in coolant flow and temperature greatly affect reactor
power and vice versa. Additionally, \gls{MSR} simulation software must include
capabilities to model the movement of delayed neutron precursors
and heat generation in the coolant; these physical phenomena are typically
absent in solid-fuel reactors.

Numerical methods for solving coupled multiphysics problems fall under two
general categories: loose coupling and tight coupling methods. Loose coupling
methods involve decoupling multiphysics problems through assumptions and/or
solving each individual set of physics separately, while tight
coupling methods involve solving the coupled sets of physics simultaneously.
Loosely coupled methods for reactor simulations typically decouple the
neutronics calculations from the thermal-hydraulics calculations
\citep{wang_review_2020}. Examples for loose coupling and tight coupling methods
are the Picard iteration method and \gls{JFNK} method, respectively. Tight
coupling methods may seem more computationally
intensive but with appropriate preconditioning methods, they can be as
competitive as or faster than loose coupling \citep{wang_review_2020}.
Tight coupling methods also boast higher accuracy and better convergence rates
for some strongly coupled problems such as the coupled neutronics and
thermal-hydraulics in \glspl{MSR} \citep{lindsay_introduction_2018}. On the
other hand, loosely coupled methods benefit from the relative ease of software
implementation and the extensive pool of code validation and verification in
existing literature for well-established single-physics reactor software. Users
can also mitigate the impact on accuracy and convergence rates through
careful tuning of simulation parameters such as timestep sizes.

This paper presents benchmarking results from Moltres, a coupled
neutronics/ thermal-hydraulics simulation software for \glspl{MSR}. Moltres is
built on the \gls{MOOSE} \citep{gaston_physics-based_2015} finite element
framework which provides various tools for performing fully-coupled,
NEWTON-based solves. Software benchmarking exercises are important as they
provide a common basis of comparison between software developed by different
groups of people for solving the same or similar computational problems.
Code-to-code verification of new software helps to build confidence in the
software among researchers and other stakeholders. This trust promotes
research collaboration and advances the study of more complicated problems that
feature the same underlying physics/mechanisms we test in benchmarks.
