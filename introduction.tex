\section{Introduction} \label{sec:intro}

\glspl{MSR} have attracted increasing interest over the past two
decades with various ongoing research programs funding \gls{MSR} development
in the EU \cite{cordis_severe_nodate}, China \cite{dai_17_2017}, and the US
\cite{doe_office_2021}. These programs include efforts to develop
reactor analysis software specifically tuned for \glspl{MSR}. Reactor
analysis software inform design choices that
maximize safety, reduce proliferation risks, and improve fuel
efficiency in the next generation of advanced reactors.

Liquid-fueled \glspl{MSR} present new challenges in computational reactor
safety analysis arising from fuel movement. \glspl{MSR} feature strong
negative temperature reactivity feedback in the primary fuel salt, leading to
strong and near-instantaneous
interactions between reactor power and thermal-hydraulics. Thus,
unexpected changes in coolant flow and temperature significantly affect reactor
power and vice versa, requiring tighter coupling between these physics
in the software. Additionally, \gls{MSR} simulation software must include
capabilities to model the movement of delayed neutron precursors
and heat generation in the fuel salt\textemdash physical phenomena
absent in solid-fuel reactors. The movement of delayed neutron precursors
impacts the effective delayed neutron fraction in the core and consequently
also impacts the transient behavior of the reactor. At the same time, the fuel
salt serves a dual role of generating fission heat and cooling the reactor
core, leading to stronger coupling between heat generation and salt flow
dynamics than traditional \glspl{LWR}.

\gls{MSR} research in the early 2000s led to the development of several novel
\gls{MSR} simulation tools such as Cinsf1D \cite{lecarpentier_neutronic_2003},
SIMMER-III \cite{rineiski_kinetics_2005}, DYN1D-MSR
\cite{krepel_dyn1d-msr_2005}, and DYN3D-MSR \cite{krepel_dyn3d-msr_2007}. These
works influenced many subsequent developments contributing to the more advanced
\gls{MSR} simulation tools available today.

Some of these simulation tools employ \textit{tight coupling} to couple
separate single-physics neutronics and thermal-hydraulics software. For
instance, researchers at the \gls{TUD} coupled the 3D neutron diffusion
software DALTON \cite{boer_validation_2010} and the \gls{CFD} software HEAT
\cite{de_zwaan_static_2007} to perform a safety analysis of the \gls{MSFR}
\cite{fiorina_modelling_2014}. In a later effort from the same institute,
Tiberga et al. \cite{tiberga_discontinuous_2019} coupled PHANTOM-$S_N$ and
DGFlows in their participation in the CNRS benchmark study
\cite{tiberga_results_2020}. Developed at the \gls{CNRS}, the CNRS benchmark
facilitates code-to-code verification of
\gls{MSR} multiphysics software \cite{aufiero_testing_2018}. Another
multiphysics package was developed at the Paul Scherrer Institute (PSI)
coupling the thermal-hydraulics system software \gls{TRACE}
\cite{nrc_trace_2007} with the
nodal neutron diffusion software \gls{PARCS} \cite{downar_parcs_2010} for the
safety analysis of the \gls{MSFR} \cite{pettersen_coupled_2016}. Coupling
single-physics software to form an integrated multiphysics solver allows
researchers to leverage older, well-validated, single-physics software.
These single-physics software are also highly optimized for solving specific
types of \glspl{PDE} relevant to the investigated system.

With modern advancements in computing hardware and growing access to
high-performance computing systems, others have developed multiphysics solvers
by coupling the \gls{CFD} software OpenFOAM
\cite{the_openfoam_foundation_ltd_openfoam_2021} with the Monte Carlo particle
transport software
Serpent \cite{leppanen_serpent_2014}, thus achieving high-fidelity neutronics
calculations in transient reactor analyses.
To that end, Serpent has a multiphysics interface supporting coupling
with other physics software \cite{leppanen_development_2013}.
Laureau et al. \cite{laureau_transient_2017}
separately developed another technique called the
\gls{TFM} method through the introduction of additional time-dependence
operators to conventional fission matrices typically used to accelerate source
convergence in Monte Carlo neutronics calculations. The \gls{TFM} method
pre-calculates three \glspl{TFM} of the reactor system in Serpent and
interpolates the matrix values during the actual transient calculations to
incorporate the effects of temperature-induced density change and Doppler
effect on the neutron cross sections and ultimately the neutron flux. Blanco
\cite{blanco_neutronic_2020} took a more integrated approach by
compiling Serpent as an internal \texttt{C}-based function within OpenFOAM's
\texttt{C++}-based framework. This approach reduced the required data
transfers between Serpent and OpenFOAM as both software have access to shared
memory during runtime. Their integrated solver employed the Quasi-Static
method for transient neutronics calculations and ran Serpent Monte Carlo
calculations several times per timestep until the solver reached convergence.

Another \gls{MSR} simulation approach involves developing ``all-in-one''
multiphysics software which handle all multiphysics calculations and data
transfer internally. Among earlier efforts, Nicolino et al.
\cite{nicolino_coupled_2008} and Zhang et al. \cite{zhang_development_2009}
recognized the
need for more robust multiphysics coupling techniques and higher-fidelity
thermal-hydraulics solutions to accurately capture complex flow profiles in
pool-type \glspl{MSR}. They each independently developed unnamed multiphysics
solvers and demonstrated them with non-moderated \gls{MSR}
designs. Later, Li et al. \cite{li_transient_2015} demonstrated the
steady-state and transient analysis capabilities of COUPLE, a neutronics and
thermal-hydraulics software developed at the Karlsruhe Institute of Technology.
Others adopted extensible software frameworks for developing numerical solvers
to develop multiphysics reactor analysis software. Examples of these software
frameworks include the commercial COMSOL
Multiphysics\textsuperscript{\textregistered} software
\cite{comsol_ab_comsol_nodate}, the aforementioned open-source CFD toolbox
OpenFOAM, and the open-source finite-element
framework \gls{MOOSE} \cite{gaston_physics-based_2015}. Researchers at
\gls{PoliMi} developed a \gls{MSR} simulation tool in COMSOL and
modeled the \gls{MSBR} as a single axisymmetric fuel channel with a uniform
flow profile \cite{cammi_multi-physics_2011}, followed by the \gls{MSRE} core
also as a single axisymmetric fuel channel with parabola-shaped laminar flow
\cite{cammi_dimensional_2012}. They later expanded on their approach by
modeling the \gls{MSRE} upper plenum, downcomer, lower plenum, primary heat
exchanger, and secondary heat exchanger as 0D systems (lumped-parameter
models), and substituting the 2D fuel channel with a 3D fuel channel which more
closely resembled the actual fuel channels in the \gls{MSRE}
\cite{zanetti_geometric_2015}. Beyond graphite-moderated \glspl{MSR}, they
also modeled the \gls{MSFR} in the same publication that featured \gls{TUD}'s
DALTON + HEAT coupled multiphysics solver.

Other institutes have dedicated significant development work towards
OpenFOAM-based \gls{MSR} simulation tools. Aufiero et al.
\cite{aufiero_development_2014} first introduced an OpenFOAM model developed
at \gls{PoliMi}. Their model implemented a neutron diffusion model and a
\gls{RANS}-based turbulence model with incompressible flow to demonstrate 2D
and 3D transient analyses of the \gls{MSFR}. Later advancements in the
\gls{PoliMi} solver include a fuel compressibility model with helium bubble
tracking to study fuel compressibility effects
\cite{cervi_development_2019} and a $SP_3$ neutron transport
model for improved neutronics calculations \cite{cervi_development_2019-1} in
the \gls{MSFR}. GeN-Foam is another OpenFOAM-based tool developed by Fiorina
et al. \cite{fiorina_gen-foam_2015} as a general reactor multiphysics solver
applicable to \glspl{MSR} and other reactor types. GeN-Foam features neutron
diffusion, $SP_3$, and $S_N$ neutronics models 
\cite{fiorina_development_2016,fiorina_gen-foam_2015,fiorina_detailed_2019},
and thermo-mechanical modeling for reactor expansion effects. Using GeN-Foam,
Altahhan et al. \cite{altahhan_preliminary_2020} developed and optimized a
liquid-fuel \gls{MSR} design while Shi \& Fratoni \cite{shi_gen-foam_2021}
benchmarked precursor drift effects in an \gls{MSRE} model.

Finally, within MOOSE, simulation tools capable of modeling
\glspl{MSR} include: Griffin \cite{wang_rattlesnake_2021}; and Moltres
\cite{lindsay_moltres_2017}\textemdash the subject of this work.
Griffin primarily tackles radiation transport problems, but the MOOSE
framework facilitates multiphysics coupling with MOOSE-based applications for
other physics such that all applications share the same data structure. This
feature eliminates
computational costs from external data transfers and optionally allows for
\textit{fully coupled} solves in which the application solves all physics
simultaneously. Similarly, Moltres benefits from the highly-integrated
cross-compatibility
within the ecosystem of MOOSE-based applications. Abou-Jaoude et al.
\cite{abou-jaoude_coupled_2020} coupled Griffin with Pronghorn, another
MOOSE-based application for advanced reactor thermal-hydraulics modeling, to
demonstrate several steady-state \gls{MSR} simulation capabilities defined in
the CNRS benchmark. Lindsay et al.
\cite{lindsay_introduction_2018} first demonstrated Moltres' \gls{MSR} modeling
capabilities on 2D axisymmetric and 3D Cartesian models of the \gls{MSRE} with
fixed velocity flow on a fully coupled neutronics and thermal-hydraulics solve.
We later demonstrated some of Moltres' more recent developments through
modeling a 2D axisymmetric model of the \gls{MSFR} for steady-state operation
and transient accident analysis \cite{park_advancement_2020}. The latter study
introduced \gls{DNP} flow looping, \gls{DNP} drift and temperature 
advection-diffusion coupling to incompressible flow, and decay heat modeling
capabilities.

The mutual compatibility among different physics applications within the
\gls{MOOSE} framework simplifies the work required to strongly couple
different physics together to solve novel multiphysics problems. For \gls{MSR}
simulations in Moltres such as those in this study, we coupled Moltres'
\gls{MSR} modeling capabilities with \gls{MOOSE}'s \texttt{Navier-Stokes} and
\texttt{Heat} \texttt{Conduction} physics modules \cite{peterson_overview_2018}
for general thermal-hydraulics modeling.

This paper presents results of the CNRS benchmark from Moltres
to verify its capabilities for modeling fast-spectrum \glspl{MSR}. Moltres
is an open-source multiphysics simulation software for advanced reactors.
Making Moltres open-source promotes quality and participation through
transparency and ease of peer review. The source code
\cite{lindsay_moltres_2017} is available on GitHub \cite{github_build_2017}.
Moltres leverages \texttt{git} for version control, and integrated testing to
protect existing capabilities while concurrently supporting continued code
development. Moltres depends on the \gls{MOOSE} finite element framework for
its meshing and parallel, nonlinear Newton-Krylov solver capabilities.
Therefore Moltres has access to \textit{fully coupled} methods with implicit
time-stepping. \textit{Full coupling} in the context of numerical methods
refers to solving multiple equations simultaneously in a larger system of
equations. In this study, we applied full coupling for the transient
cases in Phase 2 of the CNRS benchmark.

Users also have the flexibility of separating the equations through
\textit{tight coupling} in which coupling is achieved through
fixed-point iterations. For instance, users can employ full coupling to solve
the multigroup neutron diffusion equations and the Navier-Stokes equations as
two large, separate systems of equations to handle the strong coupling within
those systems, thereby benefitting from solver stability and convergence
over fewer iterations. These two systems can then be tightly coupled via outer
fixed-point iterations to handle the relatively weak (but still objectively
strong) coupling between the neutronics and thermal-hydraulics. Other than
potentially lower compute times, this segregated approach allows users to adopt
different mesh resolutions for the neutronics and thermal-hydraulics solvers.
Alternatively, users can subcycle multiple timesteps of one system during one
timestep of the other system. This flexibility helps users handle different
spatial and temporal scales of various
time-dependent phenomena in reactors. In this study, we applied tight coupling
to run criticality calculations coupled with pseudo-transient calculations of
the buoyancy-driven salt flow for the steady-state cases in Phase 1 of the CNRS
benchmark.

For the reader's convenience, we wish to note that Moltres, GeN-Foam
\cite{fiorina_gen-foam_2015}, and the PoliMi OpenFOAM solver
\cite{aufiero_development_2014} share several common characteristics as
multiphysics solvers for reactor analysis. In terms of numerical methods, all
three solvers implement general \gls{PDE} discretization methods for solving
deterministic models on unstructured meshes and highly scalable solution
algorithms, though Moltres uses \gls{FEM} and OpenFOAM uses \gls{FVM}. They
also adopt open-source philosophies and inherit \texttt{C++}
object-oriented programming structures. Lastly, all
three solvers benefit immensely from being a part of broader ecosystems of
compatible tools, eliminating external data
transfers and facilitating solver coupling.
