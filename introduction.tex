\section{Introduction} \label{sec:intro}

\glspl{MSR} have attracted increasing interest over the past two
decades with various on-going research programs funding \gls{MSR} development
in the EU \citep{cordis_severe_nodate}, China \citep{dai_17_2017}, and the US
\citep{doe_office_2021}. These programs include efforts to develop
reactor analysis software specifically tuned for \glspl{MSR}. Reactor
analysis software are important tools in reactor development because
they help inform design choices in-line with
maximizing safety, reducing proliferation risks, and improving fuel
efficiency in the next generation of advanced reactors.

Liquid-fuel \glspl{MSR} present new challenges in computational reactor
safety analysis arising from the liquid fuel form. \glspl{MSR} feature strong
negative reactivity feedback in the primary coolant which holds the dissolved
fissile material. The feedback causes strong and near-instantaneous
interactions between reactor power and thermal-hydraulics. Thus,
unexpected changes in coolant flow and temperature greatly affect reactor
power and vice versa. Additionally, \gls{MSR} simulation software must include
capabilities to model the movement of delayed neutron precursors
and heat generation in the coolant; these physical phenomena are typically
absent in solid-fuel reactors.

In recent years, several simulation tools have been developed for full-core
modeling of fast-spectrum \glspl{MSR}. One approach involves coupling
single-physics neutronics and thermal-hydraulics software via a script which
handles data transfers between the software. For example, researchers at
the Delft University of Technology coupled 3D neutron diffusion software
DALTON \citep{boer_validation_2010} and CFD software HEAT
\citep{de_zwaan_static_2007} to perform a safety analysis of the \gls{MSFR}
\citep{fiorina_modelling_2014}. In a later effort from the same institute,
Tiberga et al. coupled PHANTOM-$S_N$ and DGFlows
\citep{tiberga_discontinuous_2019} for their participation in the
CNRS Benchmark study \citep{tiberga_results_2020}. Another multiphysics
package was developed at the Paul Scherrer Institute coupling
thermal-hydraulics system software TRACE \citep{nrc_trace_2007} with the nodal
neutron diffusion software PARCS \citep{downar_parcs_2010} for the safety
analysis of the \gls{MSFR} \citep{pettersen_coupled_2016}. With modern
advancements in computing hardware and growing access to high-performance
computing, others \citep{laureau_transient_2017,blanco_neutronic_2020} have
developed multiphysics packages coupling CFD software
OpenFOAM with Monte Carlo particle transport software Serpent 2, thus allowing
for high-fidelity neutronics calculations in transient reactor analyses.

Another approach towards creating \gls{MSR} simulation tools involves
developing stand-alone multiphysics software which can perform both
neutronics and thermal-hydraulics calculations. \cite{li_transient_2015}
demonstrated the steady-state and transient analysis capabilities of COUPLE, a
neutronics and thermal-hydraulics software developed at the Karlsruhe Institute
of Technology. Others opted to leverage general multiphysics software platforms
such as COMSOL \citep{comsol_ab_comsol_2018}, OpenFOAM
\citep{openfoam_openfoam_2021}, and MOOSE \citep{gaston_physics-based_2015}.


Numerical methods for solving coupled multiphysics problems fall under two
general categories: loose coupling and tight coupling methods. Loose coupling
methods involve decoupling multiphysics problems through assumptions and/or
solving each individual set of physics separately, while tight
coupling methods involve solving the coupled sets of physics simultaneously.
Loosely coupled methods for reactor simulations typically decouple the
neutronics calculations from the thermal-hydraulics calculations
\citep{wang_review_2020}. Examples for loose coupling and tight coupling methods
are the Picard iteration method and \gls{JFNK} method, respectively. Tight
coupling methods may seem more computationally
intensive, but with appropriate preconditioning methods they can be 
competitive with loose coupling \citep{wang_review_2020}.
Tight coupling methods also boast higher accuracy and better convergence rates
for some strongly coupled problems such as the coupled neutronics and
thermal-hydraulics in \glspl{MSR} \citep{lindsay_introduction_2018}. On the
other hand, loosely coupled methods benefit from the relative ease of software
implementation and the extensive pool of code validation and verification in
existing literature for well-established single-physics reactor software. Users
can also mitigate the impact on accuracy and convergence rates through
careful tuning of simulation parameters such as timestep sizes.

This paper presents benchmarking results from Moltres, a coupled
neutronics/thermal-hydraulics simulation software for \glspl{MSR}. Moltres is
built on the \gls{MOOSE} \citep{gaston_physics-based_2015} finite element
framework which provides various tools for performing fully-coupled,
Newton-Krylov calculations. Software benchmarks are important as
they provide a common basis of comparison between different software designed
to solve similar computational problems within the same application areas.
Code-to-code verification of new software helps to build confidence in the
software among researchers and other stakeholders. This trust promotes
research collaboration and advances the study of more complicated problems with
the same underlying physics tested in the benchmarks.
