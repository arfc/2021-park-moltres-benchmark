\section{Introduction} \label{sec:intro}

\glspl{MSR} have attracted increasing interest over the past two
decades with various on-going research programs funding \gls{MSR} development
in the EU \citep{cordis_severe_nodate}, China \citep{dai_17_2017}, and the US
\citep{doe_office_2021}. These programs include efforts to develop
reactor analysis software specifically tuned for \glspl{MSR}. Reactor
analysis software are important tools in reactor development because
they help inform design choices in-line with
maximizing safety, reducing proliferation risks, and improving fuel
efficiency in the next generation of advanced reactors.

Liquid-fuel \glspl{MSR} present new challenges in computational reactor
safety analysis arising from the liquid fuel form. \glspl{MSR} feature strong
negative reactivity feedback in the primary coolant which holds the dissolved
fissile material. The feedback causes strong and near-instantaneous
interactions between reactor power and thermal-hydraulics. Thus,
unexpected changes in coolant flow and temperature greatly affect reactor
power and vice versa. Additionally, \gls{MSR} simulation software must include
capabilities to model the movement of delayed neutron precursors
and heat generation in the coolant; these physical phenomena are typically
absent in solid-fuel reactors. The movement of delayed neutron precursors
impacts the effective delayed neutron fraction in the core and consequently
also impacts the transient behavior of the reactor.

In recent years, several simulation tools have been developed for full-core
modeling of fast-spectrum \glspl{MSR}. One approach involves coupling
single-physics neutronics and thermal-hydraulics software via a script which
handles data transfers between the software. For example, researchers at
the Delft University of Technology coupled 3D neutron diffusion software
DALTON \citep{boer_validation_2010} and CFD software HEAT
\citep{de_zwaan_static_2007} to perform a safety analysis of the \gls{MSFR}
\citep{fiorina_modelling_2014}. In a later effort from the same institute,
\cite{tiberga_discontinuous_2019} coupled PHANTOM-$S_N$ and DGFlows
in their participation in the
CNRS Benchmark study \citep{tiberga_results_2020}. Another multiphysics
package was developed at the Paul Scherrer Institute (PSI) coupling
thermal-hydraulics system software TRACE \citep{nrc_trace_2007} with the nodal
neutron diffusion software PARCS \citep{downar_parcs_2010} for the safety
analysis of the \gls{MSFR} \citep{pettersen_coupled_2016}. With modern
advancements in computing hardware and growing access to high-performance
computing, others \citep{laureau_transient_2017,blanco_neutronic_2020} have
developed multiphysics packages coupling CFD software
OpenFOAM with Monte Carlo particle transport software Serpent 2, thus allowing
for high-fidelity neutronics calculations in transient reactor analyses.

Another approach towards creating \gls{MSR} simulation tools involves
developing stand-alone multiphysics software which can perform both
neutronics and thermal-hydraulics calculations. Among earlier efforts,
\cite{nicolino_coupled_2008} and \cite{zhang_development_2009} recognized the
need for more robust multiphysics coupling techniques and higher-fidelity
thermal hydraulics solutions to accurately capture complex flow profiles in
pool-type \glspl{MSR}. They each independently developed unnamed multiphysics
simulation tools and demonstrated their tools with non-moderated \gls{MSR}
designs. Later, \cite{li_transient_2015} demonstrated the steady-state and
transient analysis capabilities of COUPLE, a neutronics and thermal-hydraulics
software developed at the Karlsruhe Institute of Technology.  Others opted to
leverage general multiphysics software platforms such as OpenFOAM
\citep{openfoam_openfoam_2021} and \gls{MOOSE}
\citep{gaston_physics-based_2015}.
Several parallel efforts at various European institutes have developed coupled
neutronics and thermal-hydraulics tools in OpenFOAM. These tools share some
similarities such as implementing the $SP_N$ simplified $P_N$ neutron transport
model and leveraging OpenFOAM's turbulent flow modeling capabilities.
Differences include the aforementioned external coupling capability with
Serpent 2 from the Centre national de la scientifique (CNRS)
\citep{blanco_neutronic_2020}, fuel compressibility modeling and helium bubble
tracking capability from Politecnico di Milano (PoliMi)
\citep{cervi_development_2019}, and fuel performance analysis capability from
PSI \citep{fiorina_creation_2018}. Within the modular MOOSE finite element
framework, \gls{MSR} simulation tools include Moltres
\citep{lindsay_introduction_2018}, the subject of this work and Rattlesnake
\citep{wang_rattlesnake_2021}, both of which rely on MOOSE's built-in
incompressible flow physics module. Rattlesnake is primarily a radiation
transport application but the MOOSE framework facilitates multiphysics coupling
with other MOOSE-based applications such as Pronghorn for thermal-hydraulics
such that all applications share the same data structure, thus eliminating
computational costs from external data transfers and optionally allowing for
all physics to be solved simultaneously instead of sequentially through
fixed-point iterations. Similarly, Moltres benefits from the highly-integrated
cross compatibility within the ``ecosystem'' of MOOSE-based applications.

%Numerical methods for solving coupled multiphysics problems fall under two
%general categories: loose coupling and tight coupling methods. Loose coupling
%methods involve decoupling multiphysics problems through assumptions and/or
%solving each individual set of physics separately, while tight
%coupling methods involve solving the coupled sets of physics simultaneously.
%Loosely coupled methods for reactor simulations typically decouple the
%neutronics calculations from the thermal-hydraulics calculations
%\citep{wang_review_2020}. Examples for loose coupling and tight coupling methods
%are the Picard iteration method and \gls{JFNK} method, respectively. Tight
%coupling methods may seem more computationally
%intensive, but with appropriate preconditioning methods they can be 
%competitive with loose coupling \citep{wang_review_2020}.
%Tight coupling methods also boast higher accuracy and better convergence rates
%for some strongly coupled problems such as the coupled neutronics and
%thermal-hydraulics in \glspl{MSR} \citep{lindsay_introduction_2018}. On the
%other hand, loosely coupled methods benefit from the relative ease of software
%implementation and the extensive pool of code validation and verification in
%existing literature for well-established single-physics reactor software. Users
%can also mitigate the impact on accuracy and convergence rates through
%careful tuning of simulation parameters such as timestep sizes.

This paper presents benchmarking results from Moltres, a coupled
neutronics and thermal-hydraulics simulation software for \glspl{MSR}. Moltres
is an open-source, \gls{MOOSE}-based application designed for multiphysics
simulations of \glspl{MSR}. The goal of making Moltres open-source is to
promote quality and participation through transparency and
ease of peer review. The source code \citep{lindsay_moltres_2017} is available
on GitHub \citep{github_build_2017}. Moltres leverages \texttt{git} for
version control, and integrated testing to protect existing capabilities while
concurrently supporting continued code development. Moltres depends on the
\gls{MOOSE} finite element framework for its meshing and parallel, nonlinear
Newton-Krylov solver capabilities. Therefore Moltres, by default, has access to
\textit{strong coupling} methods with implicit time-stepping.
\textit{Strong coupling} in the context of numerical methods refers to
solving multiple physics represented by multiple equations simultaneously.
Users also have the flexibility of separating different physics through
\textit{tight coupling} in which multiphysics coupling is achieved by
fixed-point iterations.

The mutual compatibility among different physics applications within the
\gls{MOOSE} framework simplifies the work required to strongly couple
different physics together to solve novel multiphysics problems. For \gls{MSR}
simulations in Moltres such as those in this paper, we coupled Moltres'
\gls{MSR} modeling capabilities with \gls{MOOSE}'s ``Navier-Stokes'' and
``Heat Conduction'' physics modules \citep{peterson_overview_2018} for
general thermal-hydraulics modeling.

The benchmark of interest in this work is the CNRS Benchmark
\citep{tiberga_results_2020} for fast-spectrum \gls{MSR} simulation tools.
Software benchmarks are important as
they provide a common basis of comparison between different software designed
to solve similar computational problems within the same application areas.
Rigorous code-to-code verification of reactor physics software is especially
important given the implications on reactor safety and the risks of high
radioactivity exposure in nuclear accident scenarios.
