%        File: revise.tex
%     Created: Wed Oct 27 02:00 PM 2018 P
% Last Change: Wed Oct 27 02:00 PM 2018 P
%

%
% Copyright 2007, 2008, 2009 Elsevier Ltd
%
% This file is part of the 'Elsarticle Bundle'.
% ---------------------------------------------
%
% It may be distributed under the conditions of the LaTeX Project Public
% License, either version 1.2 of this license or (at your option) any
% later version.  The latest version of this license is in
%    http://www.latex-project.org/lppl.txt
% and version 1.2 or later is part of all distributions of LaTeX
% version 1999/12/01 or later.
%
% The list of all files belonging to the 'Elsarticle Bundle' is
% given in the file `manifest.txt'.
%

% Template article for Elsevier's document class `elsarticle'
% with numbered style bibliographic references
% SP 2008/03/01
%
%
%
% $Id: elsarticle-template-num.tex 4 2009-10-24 08:22:58Z rishi $
%
%
%\documentclass[preprint,12pt]{elsarticle}
\documentclass[answers,12pt]{exam}

% \documentclass[preprint,review,12pt]{elsarticle}

% Use the options 1p,twocolumn; 3p; 3p,twocolumn; 5p; or 5p,twocolumn
% for a journal layout:
% \documentclass[final,1p,times]{elsarticle}
% \documentclass[final,1p,times,twocolumn]{elsarticle}
% \documentclass[final,3p,times]{elsarticle}
% \documentclass[final,3p,times,twocolumn]{elsarticle}
% \documentclass[final,5p,times]{elsarticle}
% \documentclass[final,5p,times,twocolumn]{elsarticle}

% if you use PostScript figures in your article
% use the graphics package for simple commands
% \usepackage{graphics}
% or use the graphicx package for more complicated commands
\usepackage{graphicx}
\usepackage{tabularx}
\usepackage{booktabs}
% or use the epsfig package if you prefer to use the old commands
% \usepackage{epsfig}

% The amssymb package provides various useful mathematical symbols
\usepackage{amssymb}
% The amsthm package provides extended theorem environments
% \usepackage{amsthm}
\usepackage{amsmath}

% The lineno packages adds line numbers. Start line numbering with
% \begin{linenumbers}, end it with \end{linenumbers}. Or switch it on
% for the whole article with \linenumbers after \end{frontmatter}.
\usepackage{lineno}

% I like to be in control
\usepackage{placeins}

% natbib.sty is loaded by default. However, natbib options can be
% provided with \biboptions{...} command. Following options are
% valid:

%   round  -  round parentheses are used (default)
%   square -  square brackets are used   [option]
%   curly  -  curly braces are used      {option}
%   angle  -  angle brackets are used    <option>
%   semicolon  -  multiple citations separated by semi-colon
%   colon  - same as semicolon, an earlier confusion
%   comma  -  separated by comma
%   numbers-  selects numerical citations
%   super  -  numerical citations as superscripts
%   sort   -  sorts multiple citations according to order in ref. list
%   sort&compress   -  like sort, but also compresses numerical citations
%   compress - compresses without sorting
%
% \biboptions{comma,round}

% \biboptions{}


\usepackage{xspace}
\usepackage{color}

\usepackage{multirow}
\usepackage[hyphens]{url}


\usepackage[acronym,toc]{glossaries}
%\newacronym{<++>}{<++>}{<++>}
\newacronym[longplural={metric tons of heavy metal}]{MTHM}{MTHM}{metric ton of heavy metal}
\newacronym{ABM}{ABM}{agent-based modeling}
\newacronym{ACDIS}{ACDIS}{Program in Arms Control \& Domestic and International Security}
\newacronym{AHTR}{AHTR}{Advanced High Temperature Reactor}
\newacronym{ANDRA}{ANDRA}{Agence Nationale pour la gestion des D\'echets RAdioactifs, the French National Agency for Radioactive Waste Management}
\newacronym{ANL}{ANL}{Argonne National Laboratory}
\newacronym{API}{API}{application programming interface}
\newacronym{ARDP}{ARDP}{Advanced Reactor Demonstration Program}
\newacronym{ARE}{ARE}{Aircraft Reactor Experiment}
\newacronym{ASME}{ASME}{American Society of Mechanical Engineers}
\newacronym{ATWS}{ATWS}{Anticipated Transient Without Scram}
\newacronym{BDBE}{BDBE}{Beyond Design Basis Event}
\newacronym{BIDS}{BIDS}{Berkeley Institute for Data Science}
\newacronym{CAFCA}{CAFCA}{ Code for Advanced Fuel Cycles Assessment }
\newacronym{CDTN}{CDTN}{Centro de Desenvolvimento da Tecnologia Nuclear}
\newacronym{CEA}{CEA}{Commissariat \`a l'\'Energie Atomique et aux \'Energies Alternatives}
\newacronym{CFD}{CFD}{Computational Fluid Dynamics}
\newacronym{CI}{CI}{continuous integration}
\newacronym{CNEN}{CNEN}{Comiss\~{a}o Nacional de Energia Nuclear}
\newacronym{CNERG}{CNERG}{Computational Nuclear Engineering Research Group}
\newacronym{CNRS}{CNRS}{Centre National de la Recherche Scientifique, the French National Centre for Scientific Research}
\newacronym{COMSOL}{COMSOL}{COMmon SOLution}
\newacronym{COSI}{COSI}{Commelini-Sicard}
\newacronym{COTS}{COTS}{commercial, off-the-shelf}
\newacronym{CSNF}{CSNF}{commercial spent nuclear fuel}
\newacronym{CTAH}{CTAHs}{Coiled Tube Air Heaters}
\newacronym{CUBIT}{CUBIT}{CUBIT Geometry and Mesh Generation Toolkit}
\newacronym{CURIE}{CURIE}{Centralized Used Fuel Resource for Information Exchange}
\newacronym{DAG}{DAG}{directed acyclic graph}
\newacronym{DANESS}{DANESS}{Dynamic Analysis of Nuclear Energy System Strategies}
\newacronym{DBE}{DBE}{Design Basis Event}
\newacronym{DESAE}{DESAE}{Dynamic Analysis of Nuclear Energy Systems Strategies}
\newacronym{DHS}{DHS}{Department of Homeland Security}
\newacronym{DFEM}{DFEM}{discontinuous finite element method}
\newacronym{DNP}{DNP}{delayed neutron precursor}
\newacronym{DOE}{DOE}{Department of Energy}
\newacronym{DMSR}{DMSR}{Denatured Molten Salt Reactor}
\newacronym{DRACS}{DRACS}{Direct Reactor Auxiliary Cooling System}
\newacronym{DRE}{DRE}{dynamic resource exchange}
\newacronym{DSNF}{DSNF}{DOE spent nuclear fuel}
\newacronym{DYMOND}{DYMOND}{Dynamic Model of Nuclear Development }
\newacronym{EBS}{EBS}{Engineered Barrier System}
\newacronym{EDZ}{EDZ}{Excavation Disturbed Zone}
\newacronym{EPA}{EPA}{Environmental Protection Agency}
\newacronym{EP}{EP}{Engineering Physics}
\newacronym{EVOL}{EVOL}{Evaluation and Viability of Liquid Fuel Fast Reactor System}
\newacronym{FCO}{FCO}{Fuel Cycle Options}
\newacronym{FCT}{FCT}{Fuel Cycle Technology}
\newacronym{FEHM}{FEHM}{Finite Element Heat and Mass Transfer}
\newacronym{FEM}{FEM}{finite element method}
\newacronym{FEPs}{FEPs}{Features, Events, and Processes}
\newacronym{FHR}{FHR}{Fluoride-Salt-Cooled High-Temperature Reactor}
\newacronym{FLiBe}{FLiBe}{Fluoride-Lithium-Beryllium}
\newacronym{FP}{FP}{fission product}
\newacronym{GDSE}{GDSE}{Generic Disposal System Environment}
\newacronym{GDSM}{GDSM}{Generic Disposal System Model}
\newacronym{Gen IV}{Gen IV}{Generation IV}
\newacronym{GENIUSv1}{GENIUSv1}{Global Evaluation of Nuclear Infrastructure Utilization Scenarios, Version 1}
\newacronym{GENIUSv2}{GENIUSv2}{Global Evaluation of Nuclear Infrastructure Utilization Scenarios, Version 2}
\newacronym{GENIUS}{GENIUS}{Global Evaluation of Nuclear Infrastructure Utilization Scenarios}
\newacronym{GIF}{GIF}{Generation IV International Forum}
\newacronym{GPAM}{GPAM}{Generic Performance Assessment Model}
\newacronym{GRSAC}{GRSAC}{Graphite Reactor Severe Accident Code}
\newacronym{GUI}{GUI}{graphical user interface}
\newacronym{HALEU}{HALEU}{high-assay low-enriched uranium}
\newacronym{HLW}{HLW}{high level waste}
\newacronym{HPC}{HPC}{high-performance computing}
\newacronym{HTC}{HTC}{high-throughput computing}
\newacronym{HTGR}{HTGR}{High Temperature Gas-Cooled Reactor}
\newacronym{IAEA}{IAEA}{International Atomic Energy Agency}
\newacronym{IEMA}{IEMA}{Illinois Emergency Mangament Agency}
\newacronym{IHLRWM}{IHLRWM}{International High Level Radioactive Waste Management}
\newacronym{INL}{INL}{Idaho National Laboratory}
\newacronym{INS}{INS}{incompressible Navier-Stokes}
\newacronym{IPRR1}{IRP-R1}{Instituto de Pesquisas Radioativas Reator 1}
\newacronym{IRP}{IRP}{Integrated Research Project}
\newacronym{ISFSI}{ISFSI}{Independent Spent Fuel Storage Installation}
\newacronym{ISRG}{ISRG}{Independent Student Research Group}
\newacronym{JFNK}{JFNK}{Jacobian-Free Newton Krylov}
\newacronym{LANL}{LANL}{Los Alamos National Laboratory}
\newacronym{LBNL}{LBNL}{Lawrence Berkeley National Laboratory}
\newacronym{LCOE}{LCOE}{levelized cost of electricity}
\newacronym{LDRD}{LDRD}{laboratory directed research and development}
\newacronym{LEU}{LEU}{low-enriched uranium}
\newacronym{LFR}{LFR}{Lead-Cooled Fast Reactor}
\newacronym{LLNL}{LLNL}{Lawrence Livermore National Laboratory}
\newacronym{LMFBR}{LMFBR}{Liquid Metal Fast Breeder Reactor}
\newacronym{LOFC}{LOFC}{Loss of Forced Cooling}
\newacronym{LOHS}{LOHS}{Loss of Heat Sink}
\newacronym{LOLA}{LOLA}{Loss of Large Area}
\newacronym{LP}{LP}{linear program}
\newacronym{LWR}{LWR}{Light Water Reactor}
\newacronym{MAGNOX}{MAGNOX}{Magnesium Alloy Graphie Moderated Gas Cooled Uranium Oxide Reactor}
\newacronym{MA}{MA}{minor actinide}
\newacronym{MCFR}{MCFR}{Molten Chloride Fast Reactor}
\newacronym{MCNP}{MCNP}{Monte Carlo N-Particle code}
\newacronym{MILP}{MILP}{mixed-integer linear program}
\newacronym{MIT}{MIT}{the Massachusetts Institute of Technology}
\newacronym{MOAB}{MOAB}{Mesh-Oriented datABase}
\newacronym{MOOSE}{MOOSE}{Multiphysics Object-Oriented Simulation Environment}
\newacronym{MOSART}{MOSART}{Molten Salt Actinide Recycler and Transmuter}
\newacronym{MOX}{MOX}{mixed oxide}
\newacronym{MPI}{MPI}{Message Passing Interface}
\newacronym{MPM}{MPM}{Multi-Physics Modelling}
\newacronym{MRPP}{MRPP}{Multiregion Processing Plant}
\newacronym{MSBR}{MSBR}{Molten Salt Breeder Reactor}
\newacronym{MSFR}{MSFR}{Molten Salt Fast Reactor}
\newacronym{MSRE}{MSRE}{Molten Salt Reactor Experiment}
\newacronym{MSR}{MSR}{Molten Salt Reactor}
\newacronym{NAGRA}{NAGRA}{National Cooperative for the Disposal of Radioactive Waste}
\newacronym{NEAMS}{NEAMS}{Nuclear Energy Advanced Modeling and Simulation}
\newacronym{NEUP}{NEUP}{Nuclear Energy University Programs}
\newacronym{NFCSim}{NFCSim}{Nuclear Fuel Cycle Simulator}
\newacronym{NGNP}{NGNP}{Next Generation Nuclear Plant}
\newacronym{NMWPC}{NMWPC}{Nuclear MW Per Capita}
\newacronym{NNSA}{NNSA}{National Nuclear Security Administration}
\newacronym{NPP}{NPP}{Nuclear Power Plant}
\newacronym{NPRE}{NPRE}{Department of Nuclear, Plasma, and Radiological Engineering}
\newacronym{NQA1}{NQA-1}{Nuclear Quality Assurance - 1}
\newacronym{NRC}{NRC}{Nuclear Regulatory Commission}
\newacronym{NSF}{NSF}{National Science Foundation}
\newacronym{NSSC}{NSSC}{Nuclear Science and Security Consortium}
\newacronym{NUWASTE}{NUWASTE}{Nuclear Waste Assessment System for Technical Evaluation}
\newacronym{NWF}{NWF}{Nuclear Waste Fund}
\newacronym{NWTRB}{NWTRB}{Nuclear Waste Technical Review Board}
\newacronym{OCRWM}{OCRWM}{Office of Civilian Radioactive Waste Management}
\newacronym{ORION}{ORION}{ORION}
\newacronym{ORNL}{ORNL}{Oak Ridge National Laboratory}
\newacronym{PARCS}{PARCS}{Purdue Advanced Reactor Core Simulator}
\newacronym{PBAHTR}{PB-AHTR}{Pebble Bed Advanced High Temperature Reactor}
\newacronym{PBFHR}{PB-FHR}{Pebble-Bed Fluoride-Salt-Cooled High-Temperature Reactor}
\newacronym{PDE}{PDE}{partial differential equation}
\newacronym{PEI}{PEI}{Peak Environmental Impact}
\newacronym{PH}{PRONGHORN}{PRONGHORN}
\newacronym{PRIS}{PRIS}{Power Reactor Information System}
\newacronym{PRKE}{PRKE}{Point Reactor Kinetics Equations}
\newacronym{PSI}{PSI}{Paul Scherrer Institute}
\newacronym{PSPG}{PSPG}{Pressure-Stabilizing Petrov-Galerkin}
\newacronym{PWAR}{PWAR}{Pratt and Whitney Aircraft Reactor}
\newacronym{PWR}{PWR}{Pressurized Water Reactor}
\newacronym{PyNE}{PyNE}{Python toolkit for Nuclear Engineering}
\newacronym{PyRK}{PyRK}{Python for Reactor Kinetics}
\newacronym{QA}{QA}{quality assurance}
\newacronym{RANS}{RANS}{Reynolds-averaged Navier-Stokes}
\newacronym{RDD}{RD\&D}{Research Development and Demonstration}
\newacronym{RD}{R\&D}{Research and Development}
\newacronym{REE}{REE}{rare earth element}
\newacronym{RELAP}{RELAP}{Reactor Excursion and Leak Analysis Program}
\newacronym{RIA}{RIA}{Reactivity Insertion Accident}
\newacronym{RIF}{RIF}{Region-Institution-Facility}
\newacronym{ROD}{ROD}{Reactor Optimum Design}
\newacronym{SAM}{SAM}{System Analysis Module}
\newacronym{SAMOFAR}{SAMOFAR}{Safety Assessment of the Molten Salt Fast Reactor}
\newacronym{SAMOSAFER}{SAMOSAFER}{Severe Accident Modeling and Safety Assessment for Fluid-fuel Energy Reactors}
\newacronym{SFR}{SFR}{Sodium-Cooled Fast Reactor}
\newacronym{SINDAG}{SINDA{\textbackslash}G}{Systems Improved Numerical Differencing Analyzer $\backslash$ Gaski}
\newacronym{SKB}{SKB}{Svensk K\"{a}rnbr\"{a}nslehantering AB}
\newacronym{SNF}{SNF}{spent nuclear fuel}
\newacronym{SNL}{SNL}{Sandia National Laboratory}
\newacronym{STC}{STC}{specific temperature change}
\newacronym{SUPG}{SUPG}{Streamline-Upwind Petrov-Galerkin}
\newacronym{SWF}{SWF}{Separations and Waste Forms}
\newacronym{SWU}{SWU}{Separative Work Unit}
\newacronym{TFM}{TFM}{Transient Fission Matrix}
\newacronym{TMSR}{TMSR}{Thorium Molten Salt Reactor}
\newacronym{TRACE}{TRACE}{TRAC/RELAP Advanced Computational Engine}
\newacronym{TRIGA}{TRIGA}{Training Research Isotope General Atomic}
\newacronym{TRISO}{TRISO}{Tristructural Isotropic}
\newacronym{TRU}{TRU}{transuranic}
\newacronym{TSM}{TSM}{Total System Model}
\newacronym{TSPA}{TSPA}{Total System Performance Assessment for the Yucca Mountain License Application}
\newacronym{ThOX}{ThOX}{thorium oxide}
\newacronym{TUD}{TU Delft}{Technische Universiteit Delft}
\newacronym{UFD}{UFD}{Used Fuel Disposition}
\newacronym{UML}{UML}{Unified Modeling Language}
\newacronym{UOX}{UOX}{uranium oxide}
\newacronym{UQ}{UQ}{uncertainty quantification}
\newacronym{US}{US}{United States}
\newacronym{UW}{UW}{University of Wisconsin}
\newacronym{VISION}{VISION}{the Verifiable Fuel Cycle Simulation Model}
\newacronym{VVER}{VVER}{Voda-Vodyanoi Energetichesky Reaktor (Russian Pressurized Water Reactor)}
\newacronym{VV}{V\&V}{verification and validation}
\newacronym{WIPP}{WIPP}{Waste Isolation Pilot Plant}
\newacronym{YMR}{YMR}{Yucca Mountain Repository Site}
\newacronym{BOL}{BOL}{Beginning-of-Life}
\newacronym{ULOF}{ULOF}{Unprotected Loss of Flow}
\newacronym{LOSCA}{LOSCA}{Loss of Secondary Cooling Accident}
\newacronym{ULOHS}{ULOHS}{Unprotected Loss of Heat Sink}


\makeglossaries

\begin{document}

%\begin{frontmatter}

% Title, authors and addresses

% use the tnoteref command within \title for footnotes;
% use the tnotetext command for the associated footnote;
% use the fnref command within \author or \address for footnotes;
% use the fntext command for the associated footnote;
% use the corref command within \author for corresponding author footnotes;
% use the cortext command for the associated footnote;
% use the ead command for the email address,
% and the form \ead[url] for the home page:
%
% \title{Title\tnoteref{label1}}
% \tnotetext[label1]{}
% \author{Name\corref{cor1}\fnref{label2}}
% \ead{email address}
% \ead[url]{home page}
% \fntext[label2]{}
% \cortext[cor1]{}
% \address{Address\fnref{label3}}
% \fntext[label3]{}

\title{Verification of Moltres for Multiphysics Simulations of
Fast-Spectrum Molten Salt Reactors\\
        \large Response to Review Comments}
\author{Sun Myung Park, Madicken Munk}
\date{}

% use optional labels to link authors explicitly to addresses:
% \author[label1,label2]{<author name>}
% \address[label1]{<address>}
% \address[label2]{<address>}


%\author[uiuc]{Kathryn Huff}
%        \ead{kdhuff@illinois.edu}
%  \address[uiuc]{Department of Nuclear, Plasma, and Radiological Engineering,
%        118 Talbot Laboratory, MC 234, Universicy of Illinois at
%        Urbana-Champaign, Urbana, IL 61801}
%
% \end{frontmatter}
\maketitle
\section*{Review General Response}
We would like to thank the reviewers for their detailed assessment of this
paper. Your comments have resulted in changes which certainly improved the
paper.


\begin{questions}
\section*{Reviewer 1}

        %---------------------------------------------------------------------
        \question This work verified Moltres' MSR modeling capabilities against
        a multiphysics numerical benchmark developed for software dedicated to
        modeling fast-spectrum MSRs. The whole paper states clearly. Some
        comments are listed below:
        
        \begin{solution}
            We appreciate the detailed commentary that follows, which helped
            improve the paper.
        \end{solution}

        %---------------------------------------------------------------------
        \question What is the calculation method of effective diffusion
        coefficient of the delayed neutron precursors Dp shown in Equation
        (10)? Please give more descriptions.
        
        Have you ever assessed how much
        influence diffusion term in
        delayed neutron precursors Equation on the distribution of delayed
        neutron precursors?

        \begin{solution}
        	Moltres does not provide any specific calculation methods for
        	$D_P$. We typically input a fixed, pre-calculated scalar value
        	for $D_P$. We named it ``effective diffusion coefficient'' because
        	we expect delayed neutron precursor diffusion to arise mainly from
        	turbulent mixing in the salt. Given that the name may be confusing
        	for readers, we changed it to ``diffusion coefficient''.
        	
        	Yes, our assessment showed that precursor diffusion produces no
        	observable change in the CNRS benchmark cases. This is expected as
        	the Schmidt number is very high ($2\times10^8$), i.e. advective
        	transport dominates over diffusive transport. We also added the
        	following text in the first paragraph of Section 3.2 to clearly
        	state that we did not model diffusion in this study:
        	
        	``Given the high Schmidt number ($2\times10^8$)
\cite{tiberga_results_2020} of the salt, we neglected the precursor diffusion
term in Equation 10 as it has no observable effect on the distribution.''
        \end{solution}

        %---------------------------------------------------------------------
        \question Can the mesh of 0.01m$\times$0.01m give sufficient accuracy?
        \begin{solution}
        	Yes, further mesh refinement by a factor of 2 produces a very small
        	increase of 0.1pcm in the reactivity. We edited the following
        	section in the first paragraph of Section 3.2 for clarity:
        	
        	``For this work, we ran the benchmark cases on a uniformly-spaced
        	mesh consisting of 200$\times$200 elements (0.01m$\times$0.01m
        	each). This was sufficient for mesh convergence as further
        	refinement to 0.005m$\times$0.005m produced a small 0.1 pcm
        	increase in the reactivity for Step 0.2.''
        \end{solution}

        %---------------------------------------------------------------------
        \question Please give more introductions about "The only exception is
        the precursor concentration variables, which we discretized using
        zeroth-order monomial shape functions and solved using a discontinuous
        finite element method (DFEM)."
        \begin{solution}
        	Thank you for pointing this out. Our motivation for using DGFEM
        	arose from the numerical oscillations described in the last
        	paragraph of Section 3.1. In conjunction with addressing comments
        	from Reviewers 2 and 3, we edited the last paragraph in Section 3.1
        	to improve clarity on our motivations for using DGFEM for the
        	precursors as follows:
        	
        	``The velocity, temperature, and delayed neutron precursor variables are all susceptible to numerical node-to-node oscillations near discontinuous boundary conditions commonly observed when resolving advection-dominated transport using continuous FEM \cite{kuhlmann_lid-driven_2018}. MOOSE's \texttt{Navier-Stokes} module provides the SUPG stabilization scheme \cite{brooks_streamline_1982} for the velocity and temperature variables to combat these oscillations. We refer readers to \cite{peterson_overview_2018} for details on the implementation of these methods in the \texttt{Navier-Stokes} module. For the delayed neutron precursor variables, we discretized them using discontinuous shape functions supported by MOOSE's DGFEM solver. The cell-centered DGFEM scheme does not produce the aforementioned numerical oscillations observed with node-centered continuous FEM.''        	
        	
        	We also edited the sentence highlighted in this comment
        	as follows:
        	
        	``The only exceptions are the precursor concentration variables,
which we discretized using piecewise constant shape functions for the 
\gls{DGFEM} solver mentioned in Section 3.1. In terms of accuracy, \gls{DGFEM}
with piecewise constant discretization is similar to first-order \gls{FVM}
because they share the same
number of degrees of freedom. We interpolated the resulting discontinuous,
cell-centered precursor values to obtain the nodal values for results
analysis.''
        \end{solution}

        %---------------------------------------------------------------------

\section*{Reviewer 2}

        \question The paper is very well written. I have a few
        comments/remarks:
        \begin{solution}
            Thank you for your comments. We appreciate your review which has
            spurred clarifying improvements to the paper.
        \end{solution}
        %---------------------------------------------------------------------
        \question I find the benchmark description overly detailed. The
        original paper contains all the details. No need to repeat everything.
        \begin{solution}
        	This was also one of our primary concerns during our writing
        	process. We tried our best to distill the most important details
        	of each benchmark step by restricting the description of each step
        	to a single short paragraph and collating important parameters into
        	a single table with smaller font, while leaving enough information
        	for readers to interpret our results without having to repeated
        	refer to the original paper. We omitted various details such
        	as the salt physical properties, fuel salt compositions, and energy
        	group boundaries which we deemed would not significantly affect
        	the comprehensibility of our results.
        \end{solution}

        %---------------------------------------------------------------------
        \question Equation 13 needs reworking. It is not correct in general for
        cases where the viscosity is the effective viscosity.
        \begin{solution}
        	Modeling incompressible flow with
        	effective viscosity contributions from turbulence or other sources
        	is a functionality that we hope
        	to introduce in the near future. For now, we have clarified our
        	statements in our paper by indicating that the governing equations
        	represent what we currently have implemented in Moltres by adding
        	the following text to Section 3.1:
        	
        	``We present the governing equations for the various physics models
        	implemented in Moltres.''
        \end{solution}

        %---------------------------------------------------------------------
        \question Why is DGFEM used for the precursors. This needs an
        explanation. Especially, I am confused why use only order 0. Is it for
        avoiding negativity? What is the formal order of
        accuracy of the total scheme due to this?
        \begin{solution}
        	Thank you for identifying this point of confusion.
        	Our motivation for using DGFEM
        	arose from the numerical oscillations described in the last
        	paragraph of Section 3.1. We made the following edits to that
        	paragraph for clarity:
        	
        	``The velocity, temperature, and delayed neutron precursor variables are all susceptible to numerical node-to-node oscillations near discontinuous boundary conditions commonly observed when resolving advection-dominated transport using continuous FEM \cite{kuhlmann_lid-driven_2018}. MOOSE's \texttt{Navier-Stokes} module provides the SUPG stabilization scheme \cite{brooks_streamline_1982} for the velocity and temperature variables to combat these oscillations. We refer readers to \cite{peterson_overview_2018} for details on the implementation of these methods in the \texttt{Navier-Stokes} module. For the delayed neutron precursor variables, we discretized them using discontinuous shape functions supported by MOOSE's DGFEM solver. The cell-centered DGFEM scheme does not produce the aforementioned numerical oscillations observed with node-centered continuous FEM.''
        	
        	Beyond their fundamental differences, zeroth-order DGFEM and
            first-order FVM are both cell-centered schemes that share the
            same order of accuracy because they share the
            same number of degrees of freedom. If we compare cell-centered
            values from both schemes, they should match within given
            tolerance values. On the same mesh, first-order
            FEM has slightly more DOFs because it is a node-centered scheme
            (e.g. a first-order n$\times$n mesh has n$^2$ cells and
            n$^2+2$n$-1$ nodes).
            Therefore, all variables (including the precursor concentrations)
            are first-order accurate. We made the following changes in the
            relevant text to highlight this:
            
            ``The only exceptions are the precursor concentration variables,
which we discretized using piecewise constant shape functions for the 
\gls{DGFEM} solver mentioned in Section 3.1. In terms of solver accuracy,
\gls{DGFEM} with piecewise constant discretization is similar to first-order
\gls{FVM} because they share the same
number of degrees of freedom. We interpolated the resulting discontinuous,
cell-centered precursor values to obtain the nodal values for results
analysis.''
        \end{solution}

        %---------------------------------------------------------------------
        \question How do you apply the DG in this case: as the
        velocity files is continuous (and probably not pointwise div-free,
        possibly leading to trouble in DG).
        \begin{solution}
            When modeling precursor advection due to flow through DGFEM,
            Moltres samples the velocity values at the cell center. This
            does not cause any significant issues because the velocity
            distribution is first-order continuous within the cell. If we
            modeled precursors as continuous variables (assuming we run a
            simpler problem which doesn't run into oscillation issues), Moltres
            would sample the velocity values at the nodes instead.
            
            In both cases, the velocities would not be exactly pointwise
            divergence-free. However, we reiterate that this does not cause
            any issues with discontinuous variables within the given tolerances
            of our simulations as shown in our results.
        \end{solution}
        %---------------------------------------------------------------------
        \question I find the results section too lengthy. Please select results
        instead of presenting everything.
        \begin{solution}
        	We have tried our best to limit the length of our results section
        	by focusing on important observations such as the sources of
        	discrepancies and similiarities in results to the TUD model
        	results. The reactivity and discrepancy values and the plots are
        	also essential as they contribute to the transparency and
        	comprehensibility of our results.
        \end{solution}

        %---------------------------------------------------------------------
        \question You state things about flexibility of the solver
        (time-discretisation particularly), but do not discuss any such things
        in the paper further.
        \begin{solution}
        	Thank you for catching this. By ``flexibility'', we were referring
        	to the capability to employ full coupling or tight coupling
        	depending on the user's needs. We realize that our wording is not
        	clear here. We edited and expanded on this discussion to improve
        	clarity as follows:
        	
        	``This paper presents results of the CNRS benchmark from Moltres
to verify its capabilities for modeling fast-spectrum \glspl{MSR}. Moltres
is an open-source multiphysics simulation software for advanced reactors.
Making Moltres open-source promotes quality and participation through
transparency and ease of peer review. The source code
\cite{lindsay_moltres_2017} is available on GitHub \cite{github_build_2017}.
Moltres leverages \texttt{git} for version control, and integrated testing to
protect existing capabilities while concurrently supporting continued code
development. Moltres depends on the \gls{MOOSE} finite element framework for
its meshing and parallel, nonlinear Newton-Krylov solver capabilities.
Therefore Moltres has access to \textit{fully coupled} methods with implicit
time-stepping. \textit{Full coupling} in the context of numerical methods
refers to solving multiple equations simultaneously in a larger system of
equations. In this study, we applied full coupling for the transient
cases in Phase 2 of the CNRS benchmark.

Users also have the flexibility of separating the equations through
\textit{tight coupling} in which coupling is achieved through
fixed-point iterations. For instance, users can employ full coupling to solve
the multigroup neutron diffusion equations and the Navier-Stokes equations as
two large, separate systems of equations to handle the strong coupling within
those systems, thereby benefitting from solver stability and convergence
over fewer iterations. These two systems can then be tightly coupled via outer
fixed-point iterations to handle the relatively weak (but still objectively
strong) coupling between the neutronics and thermal-hydraulics. Other than
potentially lower compute times, this segregated approach allows users to adopt
different mesh resolutions for the neutronics and thermal-hydraulics solvers.
Alternatively, users can subcycle multiple timesteps of one system during one
timestep of the other system. This flexibility helps users handle different
spatial and temporal scales of various
time-dependent phenomena in reactors. In this study, we applied tight coupling
to run criticality calculations coupled with pseudo-transient calculations of
the buoyancy-driven salt flow for the steady-state cases in Phase 1 of the CNRS
benchmark.''
        \end{solution}

        %---------------------------------------------------------------------

\section*{Reviewer 3}
        %---------------------------------------------------------------------
        \question This is a well-written and interesting paper that presents in a structured and concise way the Moltres solver for MSR analysis and its verification against the CNRS benchmark. I suggest to accept the paper after addressing a few minor points.
        \begin{solution}
            Thank you for your kind comments. We appreciate your thorough
            review and suggestions which allowed us to improve this manuscript.
        \end{solution}
        %---------------------------------------------------------------------
        \question The authors should mention older (early 2000s) but important
        activities carried out during the Euratom MOST project, which led to
        the development of tools like DYN1D-MSR, DYN3D-MSR, SIMMER-III,
        Cinsf1D. This is seminal work that influenced many subsequent developments.
        \begin{solution}
        	We strongly agree, and are appreciative of the significance of
        	the earlier MSR research work
        	which lay the foundations for the MSR simulation tools we have
        	today. We added the following paragraph near the start of the
        	Introduction section:
        	
        	``\gls{MSR} research in the early 2000s led to the development of several novel
\gls{MSR} simulation tools such as Cinsf1D \cite{lecarpentier_neutronic_2003},
SIMMER-III \cite{rineiski_kinetics_2005}, DYN1D-MSR
\cite{krepel_dyn1d-msr_2005}, and DYN3D-MSR \cite{krepel_dyn3d-msr_2007}. These
works influenced many subsequent developments contributing to the more advanced
\gls{MSR} simulation tools available today.

Some of these simulation tools employ \textit{tight coupling} to couple
separate single-physics neutronics and thermal-hydraulics software. For
example, researchers at the \gls{TUD} ...''
        \end{solution}

        %---------------------------------------------------------------------
        \question OpenFOAM-based solvers are only briefly mentioned, and with
        limited references, despite a significant presence in the literature,
        and despite representing one of the most frequently used option for MSR
        investigations. The PoliMi solver and GeN-Foam also represent the MSR
        solvers that are closest to Moltres in terms of numerics (general PDE
        discretization methods on unstructured meshes and HPC-scalable solution
        algorithms,  even though they are FV instead of FEM), use of an
        object-oriented library, equations solved, and open-source philosophy.
        I would strongly recommend to extend a bit their description, maybe
        including some similarities (e.g., numerics, open-source license) and
        differences (e.g., FV vs FEM) w.r.t. Moltres, and providing a few
        references about their development and applications to both
        fast-spectrum (MSFR) and graphite-moderated MSRs (MSRE).
        \begin{solution}
        	Thank you for this suggestion. We edited the relevant paragraph to
        	incorporate more information on the PoliMi/GeN-Foam solvers as
        	follows:
        	
        	``Other institutes have dedicated significant development work towards
OpenFOAM-based \gls{MSR} simulation tools. Aufiero et al.
\cite{aufiero_development_2014} first introduced an OpenFOAM model developed
at \gls{PoliMi}. Their model implemented a neutron diffusion model and a
\gls{RANS}-based turbulence model with incompressible flow to demonstrate 2D
and 3D transient analyses of the \gls{MSFR}. Later advancements in the
\gls{PoliMi} solver include a fuel compressibility model with helium bubble
tracking to study fuel compressibility effects
\cite{cervi_development_2019} and a $SP_3$ neutron transport
model for improved neutronics calculations \cite{cervi_development_2019-1} in
the \gls{MSFR}. GeN-Foam is another OpenFOAM-based tool developed by Fiorina
et al. \cite{fiorina_gen-foam_2015} as a general reactor multiphysics solver
applicable to \glspl{MSR} and other reactor types. GeN-Foam features neutron
diffusion, $SP_3$, and $S_N$ neutronics models 
\cite{fiorina_development_2016,fiorina_gen-foam_2015,fiorina_detailed_2019},
and thermo-mechanical modeling for reactor expansion effects. Using GeN-Foam,
Altahhan et al. \cite{altahhan_preliminary_2020} developed and optimized a
liquid-fuel \gls{MSR} design while Shi \& Fratoni \cite{shi_gen-foam_2021}
benchmarked precursor drift effects in an \gls{MSRE} model.''
            
            We also added a paragraph at the end of the introduction section to
            highlight the similarities and differences:
            
            ``For the reader's convenience, we wish to note that Moltres, GeN-Foam
\cite{fiorina_gen-foam_2015}, and the PoliMi OpenFOAM solver
\cite{aufiero_development_2014} share several common characteristics as
multiphysics solvers for reactor analysis. In terms of numerical methods, all
three solvers implement general \gls{PDE} discretization methods for solving
deterministic models on unstructured meshes and highly scalable solution
algorithms, though Moltres uses \gls{FEM} and OpenFOAM uses \gls{FVM}. They
also adopt open-source philosophies and inherit \texttt{C++}
object-oriented programming structures. Lastly, all
three solvers benefit immensely from being a part of broader ecosystems of
compatible tools, eliminating external data
transfers and facilitating solver coupling.''
        \end{solution}

        %---------------------------------------------------------------------
        \question The authors mention the external coupling capabilities with
        Serpent as specific to CNRS. As a matter of fact, all OpenFOAM-based
        solvers can be coupled with Serpent, since Serpent2 is shipped with a
        multi-physics interface that support OpenFOAM. Please modify.
        \begin{solution}
        	Thank you for informing us of this capability. We edited the fourth
        	paragraph to reflect this fact:
        	
        	``With modern advancements in computing hardware and growing access to
high-performance computing systems, others have developed multiphysics solvers
by coupling the \gls{CFD} software OpenFOAM
\cite{the_openfoam_foundation_ltd_openfoam_2021} with the Monte Carlo particle
transport software
Serpent \cite{leppanen_serpent_2014}, thus achieving high-fidelity neutronics
calculations in transient reactor analyses.
To that end, Serpent has a multiphysics interface supporting coupling
with other physics software \cite{leppanen_development_2013}.
Laureau et al. \cite{laureau_transient_2017}
separately developed another technique called the
\gls{TFM} method through the introduction of additional time-dependence
operators to conventional fission matrices typically used to accelerate source
convergence in Monte Carlo neutronics calculations.''
        \end{solution}

        %---------------------------------------------------------------------
        \question Speaking of fuel performance capabilities, the authors
        mention a work from EPFL/PSI about inclusion of FRED. They may want to
        also mention a (probably more relevant) piece of work from Scolaro et
        al., who
        developed a full-fledged fuel behavior tool that can be coupled with
        any OpenFOAM solvers.
        \begin{solution}
        	Thank you for bringing this work to our attention. We expanded the
        	OpenFOAM descriptions in response to your previous comment. In
        	that process, we decided to exclude the references to fuel
        	performance codes as they were not relevant for liquid-fuel MSRs.
        	We will keep this reference in mind for future publications.
        \end{solution}

        %---------------------------------------------------------------------
        \question Throughout the paper, the authors stress the possibility of fully coupled simulations. Although this strategy has been strongly advocated in recent years by a part of our community, operator-splitting is often proposed to overcome the limitations of fully-coupled approaches (difficult preconditioning, necessity for solving for all physics at each time step, etc.).  I would suggest to adopt a more neutral wording when describing this possibility, since operator-splitting often turns out to be preferable in practical applications, notably when CFD is involved.

        \begin{solution}
        	Thank you for these comments. We agree with your view that
        	operator-splitting is better in many use cases, especially for
        	practical applications which involve multiphysics on different
        	spatial and time scales. In fact, as mentioned in Section 3.2,
        	our multiphysics simulations in
        	Phase 1 of the benchmark used segregated solvers to get the
        	thermal-hydraulics solution via pseudo-transient calculations while
        	separately performing the neutronics criticality calculations. We
        	adopted the following edits to reflect a more neutral stance on
        	full coupling
        	vs operator-splitting approaches, and instead bring more focus on
        	the flexibility that Moltres affords with access to both methods:
        	
        	``This paper presents results of the CNRS benchmark from Moltres
to verify its capabilities for modeling fast-spectrum \glspl{MSR}. Moltres
is an open-source multiphysics simulation software for advanced reactors.
Making Moltres open-source promotes quality and participation through
transparency and ease of peer review. The source code
\cite{lindsay_moltres_2017} is available on GitHub \cite{github_build_2017}.
Moltres leverages \texttt{git} for version control, and integrated testing to
protect existing capabilities while concurrently supporting continued code
development. Moltres depends on the \gls{MOOSE} finite element framework for
its meshing and parallel, nonlinear Newton-Krylov solver capabilities.
Therefore Moltres has access to \textit{fully coupled} methods with implicit
time-stepping. \textit{Full coupling} in the context of numerical methods
refers to solving multiple equations simultaneously in a larger system of
equations. In this study, we applied full coupling for the transient
cases in Phase 2 of the CNRS benchmark.

Users also have the flexibility of separating the equations through
\textit{tight coupling} in which coupling is achieved through
fixed-point iterations. For instance, users can employ full coupling to solve
the multigroup neutron diffusion equations and the Navier-Stokes equations as
two large, separate systems of equations to handle the strong coupling within
those systems, thereby benefitting from solver stability and convergence
over fewer iterations. These two systems can then be tightly coupled via outer
fixed-point iterations to handle the relatively weak (but still objectively
strong) coupling between the neutronics and thermal-hydraulics. Other than
potentially lower compute times, this segregated approach allows users to adopt
different mesh resolutions for the neutronics and thermal-hydraulics solvers.
Alternatively, users can subcycle multiple timesteps of one system during one
timestep of the other system. This flexibility helps users handle different
spatial and temporal scales of various
time-dependent phenomena in reactors. In this study, we applied tight coupling
to run criticality calculations coupled with pseudo-transient calculations of
the buoyancy-driven salt flow for the steady-state cases in Phase 1 of the CNRS
benchmark.''
        \end{solution}
        
        %---------------------------------------------------------------------
        \question Since the authors have the possibility to use both approaches, I would strongly encourage them to re-run some of the fully coupled calculations they have run for the benchmark by adopting a tight-coupling segregated (operator-splitting) strategy, and briefly comment on the convergence properties of the two approaches. This would provide very valuable information for the MSR community, and possibly better justify the
frequent mentioning of the fully-coupled option in the paper.

        \begin{solution}
            Thank you for this suggestion. This would definitely benefit the
            MSR community. By comparing these
            two methods for various MSR-related simulations, we could provide
            general recommendations on which methods to employ for each
            use case. However, we feel that a comparison of the two approaches
            is out of the scope of this benchmark paper. We will look into
            this suggestion for a future work.
        	
        	Because timing and compute needs are important in choosing which
        	methods to use, we provided the compute time of our
        	Phase 2 calculations through the following edits:
        	
        	``We ran all of the simulations on Cray XE nodes on the Blue Waters
supercomputer. Each XE node comprises two AMD Opteron\texttrademark\ 6276
processors, for a total of 32 CPU cores per node, rated at a maximum clock
speed of 3.2 GHz. Lindsay et al. \cite{lindsay_introduction_2018} previously
reported good scaling performance of Moltres on the Blue Waters system.

While the simulations for Phases 0 and 1 were not computationally intensive,
some required more memory than the 8GB to 16GB typically available on most
personal computers. Table \ref{table:compute} shows the compute times required
per perturbation cycle of the highest and lowest perturbation frequencies for
Step 2.1 on 16 XE nodes. The compute times of all other simulations in Step 2.1
fall between 0.98 and 6.29 hours. The simulations with smaller perturbation
frequencies required longer compute times because the larger timestep sizes led
to greater changes in the variables per timestep. The compute times are
comparable to the compute times for the CNRS-$SP_1$ and CNRS-$SP_3$ models on
20 Intel\textsuperscript{\tiny\textregistered}
Xeon\textsuperscript{\tiny\textregistered} Gold 5118 processors reported in
\cite{blanco_neutronic_2021,blanco_neutronic_2020}.''

        Table 9: Compute times required for one perturbation cycle in Step 2.1
        for $f=0.0125$ Hz and $0.8$ Hz on 16 XE nodes (512 CPU cores).
	    
	    \begin{tabular}{l c c}
		    \toprule
		    $f$ [Hz] & 0.0125 & 0.8 \\
		    \midrule
		    Compute time [hours] & 6.29 & 0.98 \\
		    \bottomrule
	    \end{tabular}
        \end{solution}

        %---------------------------------------------------------------------
        \question At some point, the authors mention:
"On the other hand, for the delayed neutron precursor variables, we discretized them using discontinuous shape functions supported by MOOSE's discontinuous finite element solver to circumvent the numerical instability issue."
In my understanding, stability problems result from the pressure-velocity coupling, and should not affect precursors. Can the authors confirm that  they observe stability problems on the precursors equations when not using discontinuous finite elements?

        \begin{solution}
        	Thank you for identifying this point of confusion. By ``numerical
        	instability'', we are referring to node-to-node oscillations in the
        	velocity, temperature, and precursors near discontinuous
        	boundary conditions (such as the top-left corner where the zero and
        	non-zero boundary conditions coincide). This is a well-documented
        	problem for node-centered, continuous FEM solvers
        	\cite{kuhlmann_lid-driven_2018}. FVM solvers do not encounter
        	this issue due to their cell-centered implementation. We now
        	realize that readers may confuse this issue with the more commonly
        	known pressure-velocity coupling instability issue. We have made
        	the following modifications:
        	
        	``The velocity, temperature, and delayed neutron precursor variables are all susceptible to numerical node-to-node oscillations near discontinuous boundary conditions commonly observed when resolving advection-dominated transport using continuous FEM \cite{kuhlmann_lid-driven_2018}. MOOSE's \texttt{Navier-Stokes} module provides the SUPG stabilization scheme \cite{brooks_streamline_1982} for the velocity and temperature variables to combat these oscillations. We refer readers to \cite{peterson_overview_2018} for details on the implementation of these methods in the \texttt{Navier-Stokes} module. For the delayed neutron precursor variables, we discretized them using discontinuous shape functions supported by MOOSE's DGFEM solver. The cell-centered DGFEM scheme does not produce the aforementioned numerical oscillations observed with node-centered continuous FEM.''
        \end{solution}

        %---------------------------------------------------------------------
\end{questions}

\section*{Other Changes}

\begin{itemize}
    \item Minor grammatical corrections
    \item Replaced all references of ``Rattlesnake'' to its new name
    ``Griffin''
\end{itemize}

\bibliographystyle{unsrt}
\bibliography{../bibliography}
\end{document}

  %
  % End of file `elsarticle-template-num.tex'.
